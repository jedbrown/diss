Viele der weltweiten wissenschaftlichen Anstrengungen sind darauf
ausgerichtet, das Verständnis des Klimawandels und seiner Konsequenzen zu
verstehen und vorherzusagen, dies insbesondere auch im Hinblick auf die
möglicherweise katastrophalen Auswirkungen auf die Gesellschaft. Numerische
Modelle spielen eine immer wichtigere Rolle bei der Analyse komplexer
Prozesse, allerdings ist der Anwendungsbereich der Modelle limitiert durch die
gegenwärtigen Modell-Ansätze.  Mit zunehmender Modell-Komplexität ist es
schwierig, die Güte der Implementation zu verifizieren, die Genauigkeit der
Simulation zu be\-stimmen, und zwischen numerischen und Modellfehlern zu
unterscheiden.  Die etablierten Strategien der Modellkopplung, die als
notwendig zur Bewältigung der Komplexität angesehen werden, verursachen
signifikante Stabilitäts- und Genauigkeitsprobleme.  Die Effizienz von
nichtlinearen Solvern sind ein weiteres Hindernis für hohe Auflösung und
fortgeschrittene Analysetechniken wie etwa Optimierung, Quantifizierung der
Ungenauigkeit und Stabilitätsanalyse.  Heutige Implementierungen verwenden
Diskretisierungen niedriger Ordnung welche moderne Hardware nur schlecht
ausnutzt, sind von niedriger Genauigkeit, und rufen in gewissen Fällen
numerische Artefakte hervor.  Diese Arbeit enthält Beiträge zu jeder dieser
Herausforderungen.

Eine neue Perspektive auf Finite-Elemente Methoden höherer Ordnung wird
eingeführt.  Diese Formulierung ergänzt die Fortschritte bei linearen und
nichtlinearen Solvern, und nützt die Eigenschaften moderner Hardware viel
besser aus als herkömmliche Methoden.  Eine neuentwickelte, allgemein
verwendbare Programm-Bibliothek mit Namen Dohp wird präsentiert, deren
Performance um ein vielfaches besser ist als diejenige von anderen
gebräuchlichen Finite-Elemente Programm-Bibliotheken.  Die hohe Performance
wird erreicht durch neue Software-Interfaces, während gleichzeitig die
Flexibilität zur Laufzeit bei der Auswahl von Elementen und Preconditionern
grösser ist.  Die Performance des Codes steigt drastisch mit höherer
Element-Ordnung.  Diese Programm-Bibliothek stellt auch mehr geometrische
Information bereit als andere Open-Source Codes, und erlaubt dadurch eine
natürliche Kopplung mit CAD und geometrischen Modellen, sowie die implizite
Lösung von Gleichungssystemen, in denen die Modell-Geometrie Teil der Lösung
ist.

Ein neuer Newton-Krylov Multigrid-Solver für die hydrostatischen
Eisschild-Gleichungen wird dargestellt.  Der hohe Aufwand zur Lösung der
hydrostatischen Gleichungen mit konventionellen Methoden war bisher der
Hauptgrund für deren Nicht-Gebrauch in grossen Eisschildmodellen --
stattdessen werden einfachere Approximationen der Impuls-Bilanz verwendet.
Neben den schlechten Algorithmen hat die Modellier-Gemeinde auch keine gute
Implementation für Parallelcomputer zur Hand, was den Anwendungsbereich weiter
stark einschränkt.  Der neue Solver zeigt eine Multigrid-Effizienz wie aus dem
Lehrbuch für mehrere anspruchsvolle Probleme, bietet eine Laufzeitverminderung
um Grössenordnungen für interessierende Problemgrössen, und zeigt nahezu
perfekte starke und schwache Skalierbarkeit auf paralleler Hardware.

Ein neues algebraisches Interface für die Kopplung verschiedener Modell-Codes
wird eingeführt.  Die robuste Kopplung mehrerer physikalischer Prozesse ist
eine Herausforderung für welche viele der normalerweise verwendeten Methoden
inadäquat sind.  Welche Koppelungsmethode sich am besten eignet ist
problemabhängig, ändert sich mit der Anzahl der betrachteten Probleme, und
sind ein Gebiet aktiver Forschung.  Eine entscheidende Einschränkung
bisheriger Software war, dass das Ausprobieren verschiedener Methoden
üblicherweise einen grossen Programmieraufwand des Benutzers erforderte.  Die
mangelnde Unterstützung durch die Software erschwerte das Testen von Qualität
und Performance verschiedener Methoden, und führte oft zur Verwendung
inadäquater Methoden für die zu lösenden Probleme.  Das neue algebraische
Interface erlaubt es, eine beliebige Zahl physikalischer Prozesse mit einer
grossen Auswahl an Methoden zu koppeln, welche zur Laufzeit ausgewählt und
kombiniert werden können.  Es erlaubt die unveränderte Verwendung bestehender
Programmmodule für die einzelnen Prozesse, und bietet daher bessere
Unterstützung von Verifikation und Extensibilität.  Das Interface bietet eine
höhere Performance und weitaus mehr Flexibilität bei der Auswahl der Methoden
als bisherige Software.  Diese Software, zusammen mit impliziten
Zeitintegratoren für differentiell-algebraische Gleichungen sowie optimaler
expliziter, stark stabilitätserhaltender Integration für hyperbolische
Systeme, wurde in PETSc implementiert und ist bei mehreren externen
Forschungsgruppen in Gebrauch.

Verbesserungen des Durchsatzes auf moderner Hardware werden
dargestellt. Bisherige Methoden zur Lösung partieller Differentialgleichungen
zeigen eine schlechte Ausnützung moderner Hardware, oft unter 5 Prozent, wegen
der überwiegenden Abhängigkeit von der Memory-Bandbreite.  Zum Teil war diese
schlechte Ausnützung auf Implementationsprobleme der ``Sparse Matrix Kernels''
zurückzuführen, die zu schlechter Auslastung des high-level Caches führten.
Im Rahmen dieser Arbeit wurden die PETSc Sparse Matrix Kernels um 20 bis 30
Prozent verbessert -- die Performance ist nun nahe der theoretischen Grenze
der Hardware.  Der fundamentaleren Limitierung von Memory-Bandbreite lässt
sich nicht mit einer optimierten Implementierung beikommen, dafür sind
Veränderungen des Lösungs-Algorithmus notwendig.  Im Kontext der
Finite-Elemente-Bibliothek Dohp kann das durch Vermeiden von assemblierten
``Sparse-Matrices'' erreicht werden, indem der Matrix-freien Repräsentation
Vorzug gegeben wird, welche eine höhere arithmetische Intensität und stark
reduzierte Speicheranforderungen für alle Elemente, ausser jener der
niedrigsten Ordnung, hat.  Diese Transformation erlaubt eine Verbesserung um
eine Grössenordnung im Hardware-Gebrauch, und ist in Dohp für den Benutzer
transparent anwendbar.  Eine verbesserte Unterstützung für nicht-assemblierte
Repräsentationen wurde auch im Multi-Physik-Interface integriert.

Die Robustheits- und Genauigkeitsanforderungen für Eisfliessen schränken die
Auswahl der Diskretisierung und die Behandlung der Randbedingungen ein.  Viele
dieser technischen Anforderungen sind in der Glaziologie-Literatur nicht
dokumentiert, und behindern gegenwärtige Anstrengungen einer robusten
Simulation.  Diese technischen Anforderungen werden untersucht und Folgerungen
gezogen, die praktische Konsequenzen für diese Arbeit, sowie auch für künftige
Entwicklungen der Methoden für die Modellierung des Eisfliessens haben.

Gegenwärtige Formulierungen für polythermales Eis vernachlässigen die
Dichtevariation durch die Schmelze, und verursachen einen Erhaltungs-Fehler
erster Ordnung im Schmelzanteil.  Eine neue Kontinuums-Formulierung mit
exakter Erhaltung von Masse, Impuls und Energie, unabhängig vom Schmelzanteil,
wird präsentiert.  Für dieses System wird eine Diskretisierung hoher Ordnung
vorgeschlagen, und die numerische Genauigkeit wird mit ``Manufactured
Solutions'' untersucht.  Diese Formulierung behandelt alle Terme,
einschliesslich des Energietransportes, implizit in der Zeit, was die
Verwendung von Newton-Krylov Methoden zur Berechnung von stationären Zuständen
erlaubt.  Solche stationäre Zustände sind nützlich zur Inversion von
Parametern, ``spin up'', sowie zur Stabilitätsanalyse.  Sie werden
üblicherweise mit direkter Zeitintegration berechnet mit einer Schrittweite
die durch die CFL Stabilitätsbedingung begrenzt ist.  Diese Bedingung verlangt
üblicherweise eine sehr grosse, netzabhängige Anzahl von Zeitschritten zur
Erreichung eines stationären Zustandes.  Die Newton-Krylov Methode hingegen
konvergiert nach einer kleinen, netzunabhängigen Anzahl Iterationen.  Dieser
Solver wird für einen Ausschnitt des Jakobshavn Isbr{\ae} angewendet.  Ein
solches Modell eines Outlet-Gletschers mit realistischer Geometrie und
Randbedingungen zu erstellen ist zeitaufwendig, insbesondere wenn ein
geometrisches Modell für die Darstellung der Gleit-Randbedingung benötigt
wird, oder wenn das Netz zur Grounding Line konform sein soll.  Die
Visualisierung der Resultate wird erschwert durch die Notwendigkeit einer
Georeferenzierung.  Diese Schwierigkeiten wurden teilweise dadurch gelöst,
dass der Code mit georeferenzierten Eingabedaten in beliebigem Format und
Projektion auskommt, und eine georeferenzierte Ausgabe erzeugt.

Das Ziel dieser Arbeit ist nicht, genaue Vorhersagen zu machen, sondern die
Methoden und Prozesse für zukünftige Vorhersage-Modelle zu verbessern -- dies
insbesondere für Glaziologen die weniger an numerischen und
implementations-bedingten Problemen interessiert sind. 
