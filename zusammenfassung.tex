Ein großer Teil der wissenschaftlichen Anstrengungen weltweit auf das Verständnis und die Vorhersage der Voraus und Folgen des Klimawandels aufgrund seiner möglicherweise katastrophalen Auswirkungen auf die menschliche Gesellschaft konzentriert. 
Numerische Modelle spielen eine immer größere Rolle bei der Analyse komplexer Prozesse, sondern aktuelle Modellansätze sind Einschränkung des Umfangs der Probleme, die angesprochen werden kann. 
Mit der stetig steigenden Komplexität ist es schwierig, die Richtigkeit der Umsetzung zu überprüfen, zu bewerten Genauigkeit der Simulation, oder unterscheiden zwischen numerischen und Modellierung Fehler. 
Die etablierten Strategien für Modell-Kupplung, während allgemein angenommen werden, um Komplexität zu managen, erhebliche Stabilität und Genauigkeit Probleme notwendig. 
Effizienz der nichtlinearen Solver stellen ein weiteres Hindernis für eine hohe Auflösung und erweiterte Analyse-Techniken wie Optimierung, Unsicherheit Quantifizierung und Stabilität Analyse. 
Gegenwärtige Implementierungen neigen auch dazu, low-order Diskretisierungen, die schlecht zu nutzen Emerging-Hardware verwenden, sind niedrige Genauigkeit und verursachen numerische Artefakte in einigen Fällen. 
Diese These enthält Beiträge zu jedem dieser Herausforderungen. 
Software Beiträge umfassen eine neue Programmbibliothek für höherwertige Finite-Elemente-Methoden mit nativer Unterstützung für Code Nachprüfbarkeit während des gesamten Entwicklungszyklus, und zahlreiche Features in der {\PETSc} Programmbibliothek, insbesondere im Hinblick auf implizite Solver für fest sitzen gekoppelt, unbestimmt und mixed-type Probleme, Schnittstellen für wartbar Entwicklung von Multiphysik-Code und Zeitintegration für differentiell-algebraischen Gleichungen und Systeme, die hohe Stabilität Eigenschaften. 
Theoretische Beiträge sind die Untersuchung der Ordnungsmäßigkeit und Diskretisierung Einschränkungen für robuste Eis Strömungssimulation und eine neue konservative Formulierung für polytherme Eis. 
Das Ziel dieser Arbeit ist nicht an eine bestimmte Vorhersage zu treffen, sondern um die Methoden und Softwarewerkzeuge für künftige Vorhersmodelleirung, insbesondere durch Glaziologen weniger um numerische und rechnerische Probleme zu verbessern. 
