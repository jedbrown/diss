Realistic bathymetry posesses little regularity, thus, at every resolution that could be used for a numerical model, the slip boundary will be ``rough''.  If the roughness is smoothed significantly, then bathymetric features such as the deep channel at Jakobshavn Isbræ will be under-resolved unless an excessively fine mesh is used.  But if a slip boundary is rough on the same scale as the mesh, it becomes critical that the discretization preserve local conservation across the interface exactly instead of merely up to some mesh-dependent truncation error.

Recall the non-Newtonian Stokes problem
\begin{align}\label{eq:slip:stokes-strong}
    -\nabla \cdot(\eta D\uu) + \nabla p - \ff &= 0 \\
    \nabla \cdot \uu &= 0
\end{align}
with nonlinear viscosity
\begin{gather}
  \eta(\gamma) = B(\theta,\dotsc)\big(\epsilon + \gamma \big)^{\frac{\mathfrak{p}-2}{2}}
\end{gather}
where $D\uu = \tfrac 1 2 \left(\nabla \uu + (\nabla \uu)^T \right)$ is the strain rate, $\gamma(D\uu) = \tfrac 1 2 D\uu \tcolon D\uu$ is the second invariant, $\mathfrak{p} = 1 + \tfrac{1}{\mathfrak{n}} \approx \tfrac 4 3$, $B$ is a hardness parameter depending on enthalpy $\theta$ and perhaps other variables (e.g. grain size, dust content, damage), and $\epsilon$ is the second invariant of a reference strain rate that provides regularization to prevent viscosity from becoming infinite.

Common boundary conditions for \eqref{eq:slip:stokes-strong} include $\uu = \bm 0$ at a frozen bed, $\eta D\uu - p\bm 1 = 0$ at the free surface, and $\eta D\uu - p\bm 1 = -\rho_w z \nn$ at the ice-ocean interface underneath an ice shelf where $\nn$ is the unit outward-facing normal.
Open boundary conditions are also needed for regional models, but should be applied in places where the shallow ice approximation is accurate, thus causing the flow to be defined by local geometry.
A final, and much more difficult boundary condition, is slip at the bed.
Slip is a Dirichlet condition on the normal component and a nonlinear Robin condition on the tangent components,
\begin{align}\label{eq:slip:bcstrong}
  \uu\cdot\nn &= \bm g_{\text{melt}}(T\uu,\dotsc) \\
  T (\eta D\uu - p\bm 1)\cdot\nn &= \bm g_{\text{slip}}(T \uu,\dotsc)
\end{align}
where $T = \bm 1 - \nn\otimes\nn$ is a projector into the tangent space.
Melt rate and basal traction generally depend on a basal hydrology model, involve spatially-variant parameters, and are coupled because sliding produces heat at a rate $T\uu\cdot(\eta D\uu - p\bm 1)\cdot\nn$.
The precise form of the sliding relation is a subject of extensive debate, but is often taken to have the form
\begin{gather*}
  \bm g_{\text{slip}}(T\uu,\theta,\cdots) = \beta_m(\theta,\cdots) \abs{T\uu}^{m-1} T\uu
\end{gather*}
where $m=1$ is Navier slip, $m=1/3$ is the popular ``Weertman sliding'' \cite{weertman1957sliding} and $m\to 0$ is the Coulomb limit.
See \cite{iverson1998ring} for empirical support of the Coulomb limit and \cite{schoof2006variational,schoof2006plastic,schoof2007isg} for analysis of the associated variational inequalities.
Some continuum models for basal hydrology are discussed in \cite{flowers2002multicomponent1,flowers2002multicomponent1,johnson2002nhg}, but basal processes are poorly understood and outside the scope of the present work, see \cite{clarke2004subglacial} for a review.

% Weak form
The strong form \eqref{eq:slip:stokes-strong} is not suitable for discussing regularity issues at boundaries or our discretization so we need the weak form which is obtained by introducing test functions $\vv,q$ and integrating by parts.
Given a Lipschitz domain $\Omega \subset \R^3$ and a nonempty open subset $\Gamma$ of the boundary $\partial\Omega$ which we identify as ``not frozen'', the problem is to find $(\uu,p) \in \bm W_D^{1,\pp}(\Omega) \times L^2(\Omega)$ such that
\begin{multline}\label{eq:slip:stokes-weak}
  \int_\Omega D\vv\tcolon \eta\bm 1 \tcolon D\uu - q\div\uu - p\div\vv - \vv\cdot\bm f \\
  - \int_\Gamma \vv\cdot (\eta D\uu - p\bm 1)\cdot\nn = 0
\end{multline}
for all $(\vv,q)\in \bm W_0^{1,\mathfrak{q}}(\Omega) \times L^2(\Omega)$, where $\mathfrak q$ satisfies $1/\mathfrak p + 1/\mathfrak q = 1$.
Readers unfamiliar with Sobolev spaces may think of $W^{1,\pp}$ simply as the space with sufficiently well-behaved first derivatives.
The subscripts in $W_D^{1,\mathfrak{q}},W_0^{1,\mathfrak{q}}$ indicate that inhomogeneous and homogenous Dirichlet boundary conditions respectively are built into the space.
That is, all components of velocity are specified in regions where the bed is frozen and normal components are specified at places where sliding may take place.
The implementation of Dirichlet conditions is somewhat different from its definition here and will be discussed later.
In general circumstances, the true solution is actually only $W^{1,\pp}$, but we always assume that there is regularization which permits us to only use Hilbert spaces.
For well-posedness, it remains to specify $(D\uu - p\bm 1)\cdot\nn$ on $\Gamma$ as an algebraic function $\bm g(\uu)$ to enforce stress conditions on tangent components at slip surfaces and on all components at free surfaces in which case $\bm g$ is independent of $\uu$.

Although it is not used directly in our work, \eqref{eq:slip:stokes-weak} corresponds to the minimization of the viscous energy $\int_\Omega \half D\uu\tcolon \eta\bm 1 \tcolon D\uu$ over the subspace where $\div\uu = 0$.
In particular, it is the first variation of the Lagrangian obtained when pressure $p$ is introduced as a Lagrange multiplier to enforce the constraint.
Boundedness of the Lagrangian requires coercivity of the viscous energy term which follows from Korn's inequality which controls $\norm{\uu}_{H^1}$ using the symmetric gradient $\norm{D\uu}_{L^2}$
and an inf-sup condition to control pressure using divergence of velocity~\cite{evans1998partial,brenner2008mathematical}.
These conditions are easily satisfied by the continuum spaces, but they place important restrictions on the discrete spaces, an issue which we revisit in Section~\ref{sssec:approximation}.

\subsubsection{Singularities in the continuum formulation}
The transition from no-slip to slip boundary conditions is exactly analogous to mode II and III fatigue in nonlinear elasticity theory, transition from no-slip to an unconstrained stress condition (e.g. floating) is the mode I case.
In the case of linear rheolgy, this is the classical inverse square root stress singularity $\sigma \sim r^{-1/2}$ in fracture mechanics~\cite{anderson2005fracture}, where $r$ is the distance from the transition, see \cite{erdogan1973two} for two bonded materials.
For nonlinear rheology, the same energy estimates~\cite{rice1968path} apply and the singularity becomes $\abs{\sigma} \sim r^{(1-\pp)/\pp}$ and $\abs{D\uu} \sim r^{-1/\pp}$ as shown by \cite{rice1968plane,hutchinson1968singular}. 
These functions are all integrable and the singularity does not pose a fundamental regularity problem for the continuum equations, but the velocity in this latter case behaves as $\abs{\uu} \sim r^{(\pp-1)/\pp}$ which is a fourth root for the typical $\pp = 4/3$, thus difficult to approximate with a polynomial basis.
Note that in reality, there is not a true singularity because of friction and plastic failure in the immediate vicinity of the transition, but micro-scale physical processes are fundamentally different, therefore the meso-scale asymptotics are most relevant when designing an approximation space.

The singularity is stronger for the heat production term $\sigma\tcolon D\uu$, of order $1/r$ regardless of rheology.
Enthalpy cannot have infinite slopes because there is always physical diffusion, but the approximation problem is more difficult because of the need to represent a ``spike'' instead of a ``kink''.
The most important case for glaciology is mode III which occurs at the margins of ice streams, the thermal structure of which was investigated in \cite{jacobson1998thermal,raymond2000energy}.

Reentrant corners in the ice domain caused by incompletely resolved bathymetry are the source of the other important singularity.
For second order elliptic equations, solutions around reentrant corners of angle $\omega > \pi$ have singularities of order $r^{\pi/\omega}\sin\frac{\pi\phi}{\omega}$ where $(r,\phi), r < 0, 0 < \phi < \omega$ are polar coordinates centered at the corner, see \cite{grisvard1985elliptic,nazarov1994elliptic}, also \cite{bacuta2003regularity} which has new sharp finite element convergence estimates.
In the strongest case $\omega\to 2\pi$, this singularity becomes $r^{1/2}$ which is the same as a crack in linear media.
Nonlinear rheology is analogous and indeed, the singularity for flow past a reentrant corner is never worse than for a transition from no-slip to free slip.

Although not considered here, viscoelastic flows have the further difficulty that the Weissenberg number blows up at viscous stress singularities, see \cite{lipscomb1987implications,davies1988reentrant,hinch1993flow,owens2002cr}.

\subsubsection{Approximation spaces}\label{sssec:approximation}
While the Banach space $W^{1,\pp}$ in \eqref{eq:slip:stokes-weak} has the correct regularity for investigating the singularities discussed in the last section, approximation spaces in the present work will always be piecewise polynomial, and since we desire a symmetric formulation, we switch to the Hilbert space $H^1$.
Such polynomial bases are a conventional and proven approach, but there are promising alternatives including the rational bases used in isogeometric analysis~\cite{hughes2005isogeometric,cottrell2009isogeometric} and physics-adapted bases in extended finite element methods~\cite{belytschko2009review,mohammadi2008extended}.

Finite element methods choose a discrete subspace $\VV_D,\QQ$ of the continuous trial space $\bm H_D^1 \times L^2$ from \eqref{eq:slip:stokes-weak}, discrete spaces for the test functions, and a way of approximately\footnote{Inexact quadrature is a ``variational crime'' \cite{brenner2008mathematical}, but nonlinear rheology produces terms that cannot reasonably be integrated exactly.} evaluating integrals.
The present work considers only Galerkin methods for which case the test and trial spaces are equivalent except for inhomogenous boundary values.
Galerkin methods for elliptic problems come with a property called Galerkin orthogonality which asserts that the error in a discrete approximation is within a constant of the minimum error within the discrete space.
In other words, the PDE is solved to within a constant of the pure approximation problem for the exact solution.
This is a powerful property and not generally available for nonsymmetric problems or non-Galerkin methods.
Error estimators, adaptivity, and uncertainty quantification for both smooth and non-smooth problems are also most mature in the Galerkin context, see \cite{ainsworth1997pee,matthies2005gml,babuska2005scm,barth2010mlmcfe}.

We consider two classes of finite element space defined on hexahedral meshes.
The first is spanned by a tensor product of 1D polynomials of degree $k \ge 1$ that have been pushed forward from the reference cube $[0,1]^3$ to the physical element.
This has a variant $\Qk k$ that is continuous between elements and a discontinuous variant $\Qkdisc k$.
The other is a $\Pkdisc k$ which is spanned by polynomials of maximum degree $k$ and is discontinuous between elements.

Stability of the Galerkin approximation depends on discrete versions of Korn's inequality and the inf-sup condition.
Korn's inequality bounds the $H^1$ norm in terms of the $L^2$ norm of symmetric gradient and is easily satisfied by $C^0$ vector-valued spaces such as $Q_k$ (Korn's inequality becomes a delicate matter for nonconforming finite element spaces which are not considered here).
The inf-sup condition is more troublesome.
Given a velocity space $\VV$ and pressure space $\QQ$, the inf-sup constant
\begin{equation}\label{eq:slip:inf-sup}
  \beta = \inf_{p\in \QQ} \sup_{\uu\in\VV} \frac{\int p\div\uu}{\norm{p} \norm{\uu}}
\end{equation}
is a measure of how well the velocity space spans the pressure space.
If $\beta$ is bounded below by a positive constant as the mesh is refined, then the finite element method will converge at an optimal rate.
More precisely, given discrete solutions $(\uu_h,p_h) \in \VV \times \QQ$ with inf-sup constant $\beta$ and exact solution $(\uu,p)$, the velocity and pressure errors satisfy the a priori estimate~\cite{brezzi1991mixed}
\begin{align}\label{eq:slip:apriori}
  \eta \norm{\uu - \uu_h}_{H^1} & \le C \left[ \frac{\eta}{\beta} \inf_{\vv \in \VV} \norm{\uu - \vv}_{H^1} + \inf_{q \in \QQ} \norm{p-q}_{L^2} \right] \\
  \norm{p - p_h}_{L^2} & \le \frac{C}{\beta} \left[ \frac{\eta}{\beta} \inf_{\vv \in \VV} \norm{\uu - \vv}_{H^1} + \inf_{q\in\QQ} \norm{p-q}_{L^2} \right] .
\end{align}
This shows that the error in the discrete solution is bounded in terms of how well the true solution can be represented in the discrete space.
For problems posed in anisotropic domains or containing thin boundary layers, the true solution is most efficiently represented by highly anisotropic meshes, therefore we desire uniform inf-sup stability with respect to aspect ratio.

For many choices of $\VV \times \QQ$, such as using the same basis functions, $\beta$ is zero meaning there is a discrete pressure mode that is untested by the velocity space, thus the Stokes system is singular.
Other choices, such as $\Qk 1-\Pkdisc 0$ have positive values of $\beta$ on most grids, but $\beta$ decays under mesh refinement so spurious pressure modes appear and convergence rates suffer~\cite{brenner2008mathematical,chapelle1993inf,babuska1997babuska}.
It is important to recognize that if the pressure space contains the piecwise constant functions, the constraint equation will force the discrete velocity field $\uu$ to be exactly divergence-free when integrated over an element $e$,
\begin{gather*}
  \int_e \div\uu = \int_{\partial e} \uu\cdot\nn.
\end{gather*}
This local conservation property is important for free surface flows, density-driven flows, and for long time integration.
Note that although it is possible to solve incompressible flow problems by introducing stabilization (e.g. residual-based~\cite{hughes1986new} or polynomial projection~\cite{dohrmann2004stabilized}), these formulations sacrifice local conservation and perform poorly on problems with sharp structure.

A particularly useful and robust element pair is $\Qk k-\Pkdisc{k-1}$ where $k \ge 2$, for which local conservation holds and the inf-sup constant is uniformly bounded with respect to mesh resolution.
For smooth solutions, this element produces velocity and pressure errors of order $k$ in $H^1$ and $L^2$ respectively, which is optimal.
Unfortunately, the inf-sup constant $\beta$ degrades proportional to $\sqrt{\epsilon}$ where $\epsilon$ is the aspect ratio of the mesh.
While this is acceptable for some fluid dynamics problems, it is unusable for those geophysical flows in which extreme aspect ratio is inherent in the problem, as well as for wall-resolved large eddy simulation where boundary layer elements have aspect ratio on the order of $10^{-6}$.
An alternative is $\Qk k - \Qkdisc{k-2}$ which is locally conservative and has uniform inf-sup stability independent of aspect ratio.
This element is more ``squishy'' inside elements than $\Pkdisc{k-1}$, and has suboptimal order of accuracy $k-1$ in $H^1$ for velocity and in $L^2$ for pressure.

\subsubsection{Implementation of Dirichlet boundary conditions}\label{sssec:implementation-dirichlet}
In the continuum context, Dirichlet boundary conditions are built into the approximation space and thus do not explicitly appear in the weak form.
There are many ways to implement Dirichlet boundary conditions in the discrete context including removal from the ansatz space, penalties, ``lifting'' the known part to the right hand side for the linear problem, zeroing rows of the Jacobian, and zeroing both rows and columns by suitable evaluation of the residual.
Most of these methods perform similarly for simple problems and simple preconditioners, but have serious deficiencies for more difficult problems and sophisticated solvers.
When possible (e.g. the Dirichlet part of the domain is not itself part of the solution such as occurs with a frozen bed or contact problem), removal of Dirichlet unknowns is a robust solution, but it introduces some complexity in managing vectors residing in the ansatz space versus vectors residing in the closure (used for output and visualization) and prevents direct addressing of neighbors in structured grid computation.
This is the standard way to enforce Dirichlet conditions in \Dohp.
Penalties are easy to implement, but a penalty parameter must be chosen which contributes to ill-conditioning which reduces the accuracy of the solution (ability to converge to very high tolerance), can contaminate Schur complements, requires the Krylov method to work in the preconditioned norm (usually means left preconditioning instead of right), and must always be paired with a preconditioner that corrects the contribution from the penalty.
``Lifting'' can be performed directly on the linear system and is performed transparently using \PETSc's \code{PCRedistribute}, but the method does not compose well with field-split and hierarchical preconditioners, and the sparse matrix manipulations require extra memory and time.
Simply replacing rows of the Jacobian matrix with rows of the identity is easy to perform independent of the element assembly loop.
Unfortunately, it destroys symmetry and pollutes the spectrum of the operator which has very problem-dependent effects on the iteration count.

Zeroing both the rows and columns corresponding to Dirichlet degrees of freedom is the best alternative when the degrees of freedom are not eliminated, but there are several implementation details to consider.
Assembling into the matrix without observing boundary conditions and zeroing afterward is not efficient with sparse matrix representations since columns are difficult to address in a compressed row format (and rows are difficult to address for compressed column formats).
A better approach is to discard contributions to those rows and columns during insertion of the element stiffness matrix and then simply place the diagonal entry afterward.
This requires a compatible residual evaluation which we now consider.

Let $\VV_D$ be a discrete ansatz space with inhomogeneous Dirichlet boundary conditions implicitly built in, $\VV_0$ be the corresponding space with homogeneous conditions, $\VV_\Gamma$ be the trace space on the Dirichlet boundary, and $\bar\VV = \VV_0 \times \VV_\Gamma$ the space of all functions in the finite element space defined on the closure of the domain.
We define three projectors on $\bar\VV$,  $R_0$ projects to the $\VV_0$ subspace of $\bar\VV$ with zero values on the boundary $\VV_\Gamma$, $R_D$ projects to the affine subspace $\VV_D$, and $R_\Gamma$ projects to the trace space $\VV_\Gamma$ with zero in the interior.
We can now write the discrete residual $F$ in terms of the ``interior'' residual $f$ that does not recognize boundary conditions as
\begin{equation}\label{eq:slip:dirichlet}
  F(u) = R_0 f(R_D u) + R_\Gamma \alpha (u - R_D 0) .
\end{equation}
The value of the possibly spatially varying scaling factor $\alpha \ne 0$ does not affect the correctness of the formulation, but weighting it to be of similar magnitude to nearby diagonal entries in the matrix (e.g. by using local viscosity and the mesh size) is preferred to improve the conditioning of the linear system including boundary conditions and more importantly when geometric multigrid is used with rediscretized (non-Galerkin) coarse level operators.
The Jacobian of \eqref{eq:slip:dirichlet} isolates the Dirichlet degrees of freedom from the rest of the system.
The implementation of \eqref{eq:slip:dirichlet} and its derivative is straightforward, local element residuals and Jacobians are evaluated with correct Dirichlet values imposed on the state $u$, the result is inserted with contributions to Dirichlet nodes discarded.
Then a loop over the boundary places the $\alpha$-scaled difference from the correct boundary values into the residual vector and inserts $\alpha$ on the diagonal of the Jacobian.

\subsubsection{Slip}
Slip boundary conditions are a combination of Dirichlet on the normal component and (usually nonlinear) Robin on the tangent component.
When the slip surface is curved, there are multiple ways to define the normal direction.
Geometric averages such as those advocated by \cite{walkley2004calculation} can be very accurate and appear to be preferable for problems with surface tension in which conservation is not essential.
When exact conservation is critical, there is no choice but to use ``conservative normals''~\cite{lynch1980finite}.
Conservative normals are defined at an arbitrary node (or mode) $i$ in which the basis function $\phi_i$ has support on the boundary $\Gamma$ by
\begin{equation}
  \label{eq:slip:conservative-normal}
  \nn_i = \frac{\int_\Gamma \phi_i \nn}{\abs{\int_\Gamma \phi_i\nn}} .
\end{equation}
If the velocity field is constrained so that $\uu_i \cdot \nn_i = 0$ for a node $i$ with support on the boundary, then node $i$ will contribute zero flux through the boundary.
We enforce this condition for all boundary nodes so the total flux is also zero across the boundary: $\int_\Gamma \uu \cdot \nn = 0$.
Note that we do not in general have that $\int_f \uu \cdot \nn = 0$ for all mesh faces $f$ for the same reason that continuous Galerkin methods are not locally conservative.
However, a weaker local conservation statement similar to that in \cite{hughes2000continuous} still holds.

There is a technical difficulty observed by \cite{walkley2004calculation} when using conservative normals with inf-sup stable spaces.
In particular, the velocity space must be at least quadratic for stability reasons and the corner basis functions of $\Pk{2}$ triangles (appearing on the surface of a tetrahedral mesh) have the property that
\begin{equation*}
  \int_\Gamma \phi_i = 0 .
\end{equation*}
This means that the normal need not be constrained on any flat element face (but having no constraint admits non-physical recirculation within elements) and that for isoparametrically mapped elements with small face curvature, the definition of the normal becomes unstable due to near cancellation and can flip direction as the surface evolves.
This problem does not occur for non-deformed quadrilaterals (on the surface of a hexahedral mesh), but may arise for sufficiently deformed elements.
The extra constraint on element quality is inconvenient for moving mesh simulations and for meshing complex structures, but it is a limitation that we accept in the present work.
An interesting alternative is to eschew Lagrange interpolants in favor of the non-negative spline bases used in isogeometric analysis~\cite{cottrell2009isogeometric}, see \cite{akkerman2010isogeometric} for recent results with free surface flows.

When implementing the normal constraint in slip boundary conditions, it is not practical to remove the normal component from the ansatz space so we prefer to leave them in as described in \secref{sssec:implementation-dirichlet}.
To enforce Dirichlet conditions on the normal component, we rotate coordinates in the global vector so that the normal component is isolated.
The rotation is undone at the element level so that local operations need not be aware that the solution vector contains rotated blocks.
This rotation affects the construction of coarse level spaces in multigrid and domain decomposition methods.
The ML~\cite{ml-guide} algebraic multigrid package allows the user to specify low-energy modes to be represented in the coarse space.
These modes should respect the change of basis so that their energies remain low.
The impact on domain decomposition methods such as FETI-DP is more delicate, see \cite{klawonn2007robust,klawonn2006dual,dohrmann2010hybrid} for details on the construction of coarse spaces.

While use of conservative normals provides exact conservation across the interface, there is no guarantee that it will preserve realistic steady states, particularly for free surface flows where the hydrostatic contribution to pressure cannot be removed.
Indeed, with a smooth curved boundary and a level surface, the momentum residual will not point in the same direction as the conservative normal.
The tangent component of the momentum residual is a spurious tangent force that causes non-physical recirculation within the fluid domain that can not be prevented simply by adding artificial friction to the slip surface.
This issue is addressed \cite{behr2004application} which we follow below.
Recall the boundary integral appearing in the weak form of the Stokes problem \eqref{eq:slip:stokes-weak}
\begin{equation}\label{eq:slip:stress-bc}
  - \int_\Gamma \vv\cdot (\eta D\uu - p\bm 1)\cdot\nn
\end{equation}
where well-posedness of the continuous weak form requires specifying the stress $(D\uu - p\bm 1)\cdot\nn$ as an algebraic function of $\uu$.
We are only concerned with the tangential part of the stress since the normal components have Dirichlet conditions imposed.
The approach of \cite{behr2004application} integrates \eqref{eq:slip:stress-bc} ``as is'', without applying any boundary condition at all.
This has the effect of extending the PDE to include the boundary.
It is completely invalid for the continuum problem, resulting in non-uniqueness since the solution to the same PDE on any extended domain with any applied boundary conditions is also a solution on the initial domain, but turns out to be valid in the discrete context.
The idea was introduced in the context of open boundary conditions for natural convection in \cite{papanastasiou1992nob} and later refined 
in the restricted context of outflow boundary conditions for advection-diffusion by \cite{griffiths1997nbc} and \cite{renardy1997inb}.
In particular, \cite{griffiths1997nbc} showed that the boundary condition produces $\bigO((h+1/\Peclet)^{p+1})$ errors in $\Linfty$ for mesh size $h$, Peclet number $\Peclet$, and finite elements of polynomial degree $p$, a result distinctly better than the $\bigO(h^{p+1} + 1/\Peclet)$ obtained for Neumann outflow conditions.
