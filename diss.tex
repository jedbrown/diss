\documentclass[11pt,pdftex]{article}

\usepackage{amsmath,amssymb,bm,graphicx,url,verbatim,slashbox,multirow,hyperref,siunitx}
\usepackage{microtype,ucs}
\usepackage[utf8x]{inputenc}

\input{JedMacros.tex}

\title{Numerical methods for ice flow modeling}
\author{Jed Brown}

\usepackage{amsmath,amssymb,bm}
\usepackage{microtype}

\usepackage{ucs}
\usepackage[utf8x]{inputenc}

%\setlength\topmargin{.25in}
\pdfadjustspacing=1

\makeatletter
% The following gives us a systematic, central way to squeeze whitespace
% around section and subsection headings.
\renewcommand\section{\@startsection {section}{1}{\z@}%
                                   {-1.6ex \@plus -1ex \@minus -.2ex}%
                                   {1.4ex \@plus.2ex}%
                                   {\normalfont\Large\bfseries}}
\renewcommand\subsection{\@startsection{subsection}{2}{\z@}%
                                     {-1.ex\@plus -1ex \@minus -.2ex}%
                                     {1.ex \@plus .2ex}%
                                     {\normalfont\large\bfseries}}
\renewcommand\subsubsection{\@startsection{subsubsection}{3}{\z@}%
                            {.5\baselineskip\@plus.7\baselineskip}%
                            {-.5em}%
                            {\normalfont\bfseries}}    
\renewcommand\paragraph{\@startsection{paragraph}{4}{\z@}%
                                    {1ex \@plus1ex \@minus.2ex}%
                                    {-1em}%
                                    {\normalfont\normalsize\slshape}}
%% \renewcommand\subsubsection{\@startsection{subsubsection}{3}{\z@}%
%%                                      {-1.ex\@plus -1ex \@minus -.2ex}%
%%                                      {1.ex \@plus .2ex}%
%%                                      {\normalfont\normalsize\bfseries}}
%% \renewcommand\paragraph{\@startsection{paragraph}{4}{\z@}%
%%                                     {1ex \@plus1ex \@minus.2ex}%
%%                                     {-1em}%
%%                                     {\normalfont\normalsize\bfseries}}
\renewcommand\subparagraph{\@startsection{subparagraph}{5}{\parindent}%
                                       {1ex \@plus1ex \@minus .2ex}%
                                       {-1em}%
                                      {\normalfont\normalsize\bfseries}}

\newcommand{\squishlist}{\vspace{-.12in}\begin{itemize}\addtolength{\itemsep}{-0.5\baselineskip}}
\newcommand{\squishend}{\end{itemize}}

\newenvironment{squeeze-item}{\vspace{-.5\baselineskip}\begin{itemize}\addtolength{\itemsep}{-0.25\baselineskip}}{\end{itemize}}

\newenvironment{lyxlist}[1]
  {\vspace*{-1em}\begin{list}{}
    {\settowidth{\labelwidth}{#1}
     \setlength{\leftmargin}{\labelwidth}
     \addtolength{\leftmargin}{\labelsep}
     \addtolength{\itemsep}{-0.25\baselineskip}
     \renewcommand{\makelabel}[1]{##1\hfil}}}
  {\end{list}}

% \renewcommand{\topfraction}{1.00}
% \renewcommand{\floatpagefraction}{1.00}
% \renewcommand{\textfraction}{0.00}
% \renewcommand{\dbltopfraction}{1.00}
% \renewcommand{\dblfloatpagefraction}{1.00}
% \makeatother

\begin{document}
\pagenumbering{roman}
\maketitle 
\thispagestyle{empty}

\pagenumbering{arabic}

\section{Introduction}
\label{sec:intro}
% A large amount of scientific effort worldwide is focused on understanding and predicting the advance and consequences of climate change, due to its potentially disastrous affects on human society.
Increasingly, the results of numerical models are being used to influence decisions regarding energy policy, water rights, property values, geoengineering projects, and many more.
Thus climate science, once an academic pursuit, is transforming into an engineering project of the grandest scale, the success of which will affect the entire planet.
However, in stark contrast to other engineering disciplines, the ``product development'' latency for climate is a human lifetime or more, observations are difficult to obtain, and experimental perturbation is nearly impossible.
This contributes to an environment in which the people creating and using numerical models never have direct feedback to assess the quality of the numerical results.
Additionally, there is no direct financial incentive to produce quality results.
Indeed, the quality of the results may never be known within the lifetime of the scientist who creates them.

The situation is very different in the industrial setting.
If a computer program is used to design a plane that malfunctions, a bridge or dam that collapses, an engine with poor efficiency, or a reservoir engineering plan that results in poor recovery, there is process of accountability.
Poor numerical results have direct financial and/or political consequences for the company or organization responsible.
In such fields, it was learned early on that the process of verification and validation~\cite{roache1998verification,babuska2004vav} is of paramount importance.
Numerical and computational issues cannot be merely an afterthought, and short cuts generally lead to incorrect results and poor understanding of complex phenomena.

This philosophy was summarized well in the 1986 Editorial Policy Statement on the Control of Numerical Accuracy for the Journal of Fluids Engineering~\cite{roache1986editorial}.
This statement was unequivocal that the time for less rigorous analysis and testing of methods had long passed, and announced that the jouarnal ``will not accept for publication any paper reporting the numerical solution of a fluids engineering problem that fails to address the task of systematic truncation error testing and accuracy estimation.''
In particular, ``a single calculation in a fixed grid will not be acceptable'' and ``the editors will not consider a reasonable agreement with experimental data to be sufficient proof of accuracy, especially if any adjustable parameters are involved.''
This policy was strengthened and extended in 1993 to its current form~\cite{jfe2004numaccuracy} which defines a list of ten criteria that must be used to assess the accuracy, robustness, efficiency, and proper documentation of a numerical method in order for it to be considered by the journal.
Many other engineering journals have since adopted similar editorial policies.
It is clear from the list, and has been confirmed by numerous colleagues in each discipline of climate modeling, that there does not exist a single climate component in any discipline that comes close to satisfying these publication criteria.
This must change if the results of numerical models are to be taken seriously in the future.

Unfortunately, careful study of spatial discretization and grid convergence is not sufficient to enable the next generation of scientific inquiry for multiphysics systems such as climate or even ice dynamics in isolation.
Time discretization and implicit solver performance must also be addressed.
As elucidated by a recent Department of Energy panel~\cite{simon2007modeling}, current models invariably rely on ``first-order accurate operator-splitting, semi-implicit and explicit time integration methods, and decoupled nonlinear solution strategies.
Such methods have not provided the stability properties needed to perform accurate simulations over the dynamical time-scales of interest.
Moreover, in most cases, numerical errors and means for controlling such errors are understood heuristically at best.''
This and a related report~\cite{washington2009scientific} prioritize further research in fast, robust linear and nonlinear solvers because these ``will directly determine the scope of feasible problems to be solved'' as, inevitably, implicit formulations and advanced analysis techniques such as optimization, uncertainty quantification, and stability and sensitivity analysis assume a central role.

Ice dynamics was identified by the fourth assessment report of the IPCC~\cite{lemk2007ar4wg1} as a crucial source of uncertainty in sea level rise estimates, with no existing models capable of simulating the physical processes responsible for the large uncertainty.
The underlying source of this uncertainty is a dynamical instability identified by \cite{weertman1974sji} and made rigorous by \cite{schoof2007isg}.
This problem of grounding line stability is fundamentally three dimensional, constant factors are important, and the overall stability is determined by multi-scale behavior such as heat flux from the ocean through thin boundary layers and small bed features that can stabilize an unstable state.
The ``full'' grounding line stability problem is on the frontier of computational science in many ways.
It involves coupling physical processes in multiple domains interacting on multiple time scales through boundary layer processes, with material and geometric anisotropy, strong nonlinearity and heterogeneity, mixed characteristic PDEs, four varieties of interacting contact problems, and uncertainty in the geometry, coefficients, and constitutive models.

Advances in geoscience simulations will come from the synergy of
\begin{itemize}
\item more accurate physical models,
\item more sophisticated mathematical algorithms, and
\item more efficient implementations of these models and algorithms that take into account the advances in computer hardware.
\end{itemize}
This synergy can only occur within a comprehensive well-thought-out software infrastructure that reflects all three facets of simulation.
This thesis contains my work on each of these topics and their synthesis.
It attempts to bring a more rigorous understanding of numerical and computational issues in ice flow modeling, with a focus on robust, extensible methods that scale to large problem sizes with efficient use of current and future hardware.
An overarching theme is the development of extensible software that can be used to solve increasingly complex problems with minimal development time, while using the best possible methods.
Much of this software has been added to the {\PETSc}~\cite{petsc-user-ref} library and is in production use by many external groups.
Components which are not part of {\PETSc} are available under a BSD-style open source license.

\chapref{chap:dohp} investigates efficient nonlinear solvers for high-order finite element methods in a rather general setting.
High order methods have accuracy benefits, but they are also capable of better utilizing modern hardware, and are necessary for certain conservation and stability properties that are necessary for robust methods on practical meshes for ice flow problems.
\chapref{chap:tme-ice} presents an especially robust and scalable multigrid method for the hydrostatic equations of ice flow.
These equations are considered to be an important intermediate description of the ice flow and were previously thought to be very expensive to solve.
Our implementation improves by several orders of magnitude on the best previous results, both in time to solution and in maximum problem size.
The code is available as a tutorial in {\PETSc}.

\chapref{chap:software} discusses several of the software components that were implemented to facilitate efficient solvers, high throughput, flexible and performant finite element methods for multi-physics problems, and designing code for easy verification.
\chapref{chap:discretization} investigates several relatively unique discretization requirements for ice flow problems.
Finally, \chapref{chap:jakobshavn} has applications to the ice stream at Jakobshavn Isbr{\ae}.


\section{Background} \label{sec:background}
% \input{background}

%    \subsection{Ice Sheet Physics and Governing Equations}
%    \label{subsec:icesheetmodeling_background}
%    \input{icesheetmodeling_background}

%    \subsection{Current Ice Sheet Models}
%    \label{subsec:current_models}
%    \input{current_models}

%    \subsection{Interoperability \& Code Frameworks}
%    \label{subsec:interoperability}
%    \input{interoperability}

%    \subsection{Solver Technology}
%    \label{subsec:solver_technology}
%    \input{solver_technology}

%    \subsection{Inverse modeling}
%    \label{subsec:background_inverse}
%    \input{background_inverse}

% \section{Proposed Research and Methods}
% \label{sec:icesheetmodeling}
% \input{icesheetmodeling}

%    \subsection{Modeling Approaches}
%    \label{subsec:modeling_approaches}
%    \input{modeling_approaches}

%       \subsubsection{3D Stokes Solver}
%       \label{subsubsec:3dstokes}
%       \input{3dstokes}

%       \subsubsection{High-Order Shallow Ice Model}
%       \label{subsubsec:high_order_shallow}
%       \input{high_order_shallow}

%       \subsubsection{Energy balance model}
%       \label{subsubsec:energy}
%       \input{energy}

%       \subsubsection{Regional Ocean Model}
%       \label{subsubsec:regional_ocean}
%       \input{regional_ocean}

%    \subsection{Enabling Technologies}
%    \label{subsec:enabling_technologies}
%    \input{enabling_technologies}

%       \subsubsection{Geometry and Mesh Generation}
%       \label{subsubsec:geometry_mesh}
%       \input{geometry_mesh}

%       \subsubsection{Solvers \& Physics-Based Preconditioning}
%       \label{subsubsec:preconditioning}
%       \input{preconditioning}

%       \subsubsection{Solution Coupling}
%       \label{subsubsec:solution_coupling}
%       \input{solution_coupling}

%       \subsubsection{Adjoint \& Inverse Methods}
%       \label{subsubsec:adjoint_inverse}
%       \input{adjoint_inverse}

%    \subsection{Model Integration}
%    \label{subsec:model_integration}
%    \input{model_integration}
\section{Efficient nonlinear solvers for nodal high-order finite element methods in 3D}\label{sec:dohp}

\section{Textbook multigrid efficiency for hydrostatic ice sheet flow}\label{sec:tme-ice}

\section{Software}\label{sec:software}
\subsection{Multiphysics coupling}\label{ssec:multiphysics}
Ice flow and more generally, earth system models, involve many physical processes occurring on multiple spatial domains and possessing multiple time scales.
An important challenge in computational science is addressing how to couple models of these processes in a maintainable way without sacrificing accuracy and analysis capability.
The traditional method of splitting in time and integrating each process independently has been shown to result in low-order splitting errors~\citep{knoll2003bat,mousseau2002inc} for many problems of interest, as well as fundamentally different results regarding the stability of the system \citep{estep2008posteriori} and existence of steady states versus limit cycle behavior~\citep{jardin20081d}.
Tightly coupled multiphysics software should enable the use of robust IMEX~\footnote{IMplicit-EXplicit (IMEX) methods treat problems with well-separated time scales by solving slow processes explicitly and fast processes implicitly. IMEX methods are designed so as not to suffer from the splitting errors that plague first-order operator splitting methods.} or fully implicit methods while permitting different physics modules to be managed independently.
There are essentially two scalable approaches to tightly coupled multiphysics: field-splitting and coupled multilevel methods such as domain decomposition or multigrid.
Field splitting is most effective when it decomposes the coupled problem into separate `blocks' of equations that are well-understood and for which known methods perform well.
Coupled multilevel methods are favored when local spectral structure and compatibility conditions of the equations are well-understood so that effective smoothers and grid transfer operators can be defined.

In many cases \citep{rannacher2000finite,jameson2001many,adams2010toward}, the monolithic coupling approach has potential to offer the best possible performance.
The general guidelines are to define the smoother in terms of low-order discretization and to smooth all components at any node or element of the grid at the same time, usually either by block Gauss-Seidel relaxation or by block incomplete LU factorization.
The former permits a nonlinear smoother which has higher arithmetic intensity, but the latter tends to be more robust, especially to strong anisotropy.
The purpose of the low-order discretization (first-order upwind for hyperbolic terms) is to retain $h$-ellipticity, a necessary and sufficient condition for the existence of a pointwise smoother~\citep{brandt1979multigrid,trottenberg2001multigrid}.
When an operator fails to satisfy $h$-ellipticity, there are high-frequency error components that are poorly corrected by the smoother.

For multi-component problems, the relative scaling between fields is important for good solver performance.
This can be difficult to achieve when different parts of the domain are in different regimes.
When the problem is indefinite, the smoother and restriction operators need to be chosen to be compatible with the inf-sup condition.
Such restriction operators are usually defined geometrically, though they have also been defined algebraically for contact problems~\citep{adams2004amm}.
The associated smoothers~\citep{vanka1986block} are relatively more expensive than for definite problems, and it is difficult to handle anisotropy because incomplete factorization as a smoother becomes problematic due to zero or negative pivots~\citep{higham2002accuracy,deniet2007solving}.

It is difficult to write generic software for coupled multigrid because many fine-grained operations, especially the definition of the local smoother, require physics- and discretization-dependent choices.
Additionally, very little theory is available for coupled multi-physics and it is unclear how to define smoothers for multi-domain problems or surface-volume coupled problems.

A different approach is to apply a Newton iteration on the coupled problem and solve the linear system with a Krylov method using a preconditioner defined in terms of some operator splitting such that the physics of each part of the decomposition can be understood and efficient solvers can be provided.
This approach~\citep{knoll2004jfn} has proven successful at reusing software, with many papers accessing such methods through {\PETSc}~\citep{petsc-web-page}.
However, until my recent additions to {\PETSc}, it was cumbersome to package separate physics separately without compromising which methods are available, and decisions about using field-split versus monolithic methods were typically required up-front.
The enhancements to {\PETSc} allow each physics to be packaged independently without sacrificing performance or runtime flexibility in choice of methods.
We first summarize a variety of field-split methods and their requirements from the user, then we examine the new linear algebraic interface by which the user can generically provide this information.

\subsection{Field-split preconditioning}\label{sec:multiphysics:fieldsplit}
Suppose the Jacobian of the original coupled problem has block structure
\begin{equation}\label{eq:fieldsplit:jacobian}
  J = \begin{pmatrix} A & B \\ C & D \end{pmatrix} .
\end{equation}
We explain the methods for $2\times 2$ block systems, but $n\times n$ block systems can be treated similarly.
Field split methods can be classified as block relaxation or factorization.

Relaxation methods are inspired by the classical stationary iterative methods the preconditioners taking the forms
\begin{align*}
  P^{-1}_{\text{Jacobi}} &= \begin{pmatrix} A & \\ & D \end{pmatrix}^{-1} \\
  P^{-1}_{\text{GS}} &= \begin{pmatrix} A & \\ C & D \end{pmatrix}^{-1} \\
  P^{-1}_{\text{SGS}} &=
      \begin{pmatrix} A & \\  & \bm 1 \end{pmatrix}^{-1}
      \left(
        \bm 1 -
        \begin{pmatrix} A & B \\ & \bm 1 \end{pmatrix}
        \begin{pmatrix} A & \\ C & D \end{pmatrix}^{-1}
      \right)
\end{align*}
which can be accessed at runtime using the {\PETSc} option \code{-pc\_fieldsplit\_type additive}, \code{multiplicative}, and \code{symmetric\_multiplicative} respectively.
Relaxation is simple to use and expected to perform well when the diagonal blocks control the dynamics of the system or in which the coupling is one-way.
Heuristically, this occurs when there are few modes $y$ in the second space for which $\norm{B y}$ is large compared to $\norm{D y}$.
We expect problems if $A$ or $D$ is nearly singular or ``small'' compared to $B$ and $C$ on a sizable subspace.
More precisely, relaxation splitting will perform well if $D$ is a good approximation to the Schur complement $S = D - CA^{-1}B$.

Stiff hyperbolic waves and systems with constraints are two common cases where relaxation breaks down.
The first is somewhat more benign and we use the shallow water equations as an illustrative example.
In conservative non-dimensional form with thickness $h$ and momentum $uh$, the flat-bed shallow water equations are
\begin{align*}
  (uh)_t + \div \Big( u\otimes uh + \frac g 2 h^2 \bm 1 \Big) & = 0 \\
  h_t + \div uh & = 0 \\
\end{align*}
where $g$ is the gravitational acceleration.
When the gravity wave speed $\sqrt{gh}$ is much faster than the velocity $u$, we are in the low Mach limit which is an especially interesting case for global circulation models.
Semi-discretizing in time using implicit Euler with step size $\Delta t$, the Jacobian takes the form
\begin{align}\label{eq:fieldsplit:swe}
  \hat J(uh,h) =
  \begin{pmatrix}
    \Delta t^{-1} \bm 1 & g h \grad \\
    \div & \Delta t^{-1} \bm 1
  \end{pmatrix}
\end{align}
where several ``slow'' terms in the second row have been suppressed.
The off-diagonal blocks carry the energy of the gravity wave and capture the stiffness that appears for large time steps, which explains why relaxation performs poorly for \eqref{eq:fieldsplit:swe}.
The alternative is to define a factorization preconditioner for \eqref{eq:fieldsplit:jacobian} as
\begin{align}\label{eq:fieldsplit:schur}
  P^{-1} & =
  \begin{pmatrix} A & B \\ & S \end{pmatrix}^{-1}
  \begin{pmatrix} 1 & \\ CA^{-1} & 1 \end{pmatrix}^{-1}
  =
  \begin{pmatrix} 1 & A^{-1} B \\  & 1 \end{pmatrix}^{-1}
  \begin{pmatrix} A & \\ C & S \end{pmatrix}^{-1}
\end{align}
where $S = D - C A^{-1} B$ is the Schur complement.
When used as a right (left) preconditioner for GMRES, the lower (upper) triangular blocks of \eqref{eq:fieldsplit:schur} can be dropped without changing the eigenvalues of the preconditioner operator~\citep{murphy2000npi,ipsen2001note} which saves one solve with $A$ per Krylov iteration.
The resulting preconditioned operator has minimal polynomial degree 2 so that GMRES converges in two iterations if the preconditioner is applied exactly.
When the problem is symmetric, it is possible to precondition MINRES with the positive definite
\begin{align*}
  P^{-1} = \begin{pmatrix} A & \\ & -S \end{pmatrix}
\end{align*}
which produces a preconditioned operator with minimal polynomial degree 3 so that MINRES converges in three iterations.
In practice, the inverses appearing in \eqref{eq:fieldsplit:schur} are only applied approximately using some scalable method such as a V-cycle of multigrid so that the outer iteration converges in a constant number of iterations independent of grid resolution.
Block factorization methods in this family are available in {\PETSc} through \code{PCFieldSplit} and can be accessed at run time using \code{-pc\_fieldsplit\_type schur -pc\_fieldsplit\_schur\_factorization\_type lower}, \code{upper}, \code{diag}, and \code{full}.

In the case of the semidiscrete shallow water equations~\eqref{eq:fieldsplit:swe}, $S$ is similar to the Helmholtz differential operator $\Delta t^{-1} - \Delta t g \div h \nabla$.
This ``good Helmholtz'' structure is characteristic of stiff wave problems and discussed at length in the review \citep{knoll2005jfn} and a variety of applications \citep{mousseau2002inc,chacon2008optimal,park2009physics}.
This Helmholtz operator is our first example of an ``auxiliary matrix'' not explicitly present in the continuum equations, but needed by the preconditioner.

The situation becomes more delicate for problems with constraints such as contact problems and incompressible flow~\citep{elman2008tcp}.
In these problems, the ``important'' part of the matrix $A$ is an elliptic operator so the Schur complement becomes dense.
For isotropic variable-viscosity Stokes problems, the Schur complement is spectrally equivalent to a mass matrix defined with respect to the inverse-viscosity weighted inner product~\citep{olshanskii2006analysis}.
This weighted mass matrix is another example of an auxiliary matrix.
Time dependence in a generalized Stokes problem with variable viscosity and density can be accommodated by using both the weighted mass matrix and a density-weighted Neumann Laplace operator~\citep{olshanskii2006uniform}.
For anisotropic viscosity, Navier-Stokes, and other indefinite systems, we no longer have provable spectral equivalence and instead turn to heuristics based on approximate commutator arguments.
The idea is to find $\tilde A$ operating on the dual space (e.g. pressure) such that
\begin{equation*}
  C A^{-1} B \approx L \tilde A^{-1} M
\end{equation*}
where $L = CM_1^{-1}B$.
Here, $M_1$ is the mass matrix in the primal space (e.g. velocity) and $M$ is a mass matrix in the dual space.
For many problems, $L$ is a discrete Laplacian with Neumann boundary conditions that can be assembled independently to preserve sparsity.
Applying such preconditioners requires solves with $M$ and $L$, but only multiplication by $\tilde A$.
There are three common approaches to constructing such an approximate commutator.

The first approximate commutator approach is to use physical arguments to define $\tilde A$ by considering the continuum operators without boundary conditions \citep[though see][]{elman2009boundary}.
An example of this is the ``pressure convection-diffusion'' preconditioner of \citet{silvester2001efficient,kay2002pss} for the Navier-Stokes equations.
Such preconditioners, if one accounts for the action of the mass matrix, require three auxiliary operators $(M,L,\tilde A)$ to be provided by the user (although approximations in terms of diagonals of existing matrices are possible).
This approach has been shown to produce nearly mesh independent convergence rates~\citep{elman2005psm,deniet2007tps,elman2008tcp} for Navier-Stokes problems, with modest dependence on Reynolds number.

The second approximate commutator approach is to use a least squares argument to define
\begin{align*}
  \tilde A = M L^{-1} C M_1^{-1} A M_1^{-1} B
\end{align*}
where $M_1$ is usually approximated by its diagonal~\citep{elman1999bfbt,elman2006bpb}.
Application of this preconditioner requires an additional solve with $L$ and consequently has higher computational cost per iteration than the first alternative, but anisotropy and other terms are naturally accommodated with no additional effort.
This method shows near mesh independence for solving the Navier-Stokes equations, and while slightly less robust, typically has better performance than the first method when Newton linearization is used~\citep{elman2008tcp}.
It has been used successfully for challenging variable viscosity Stokes problems discretized using $\Qk 1-\Pkdisc 0$ finite elements.
The ``least squares commutator'' is available in {\PETSc} using \code{PCLSC}.

The third method is to define $\tilde A$ using sparse approximate commutators~\citep{elman2006bpb}, a similar method to the sparse approximate inverse~\citep{grote1997parallel}.
This approach has the attractive property of requiring only one solve per iteration without the need for additional user-provided auxiliary matrices.

There are several important variants of the methods above including the augmented Lagrangian~\citep{awanou2005convergence,dohrmann2006pbp,deniet2007tps} which accelerates convergence by penalizing $A$ using a multiple of $B M^{-1} C$ which is singular (thus making the inner problem more difficult to solve with most methods) as well as the use of pivoting for preconditioning full-space iteration in PDE-constrained optimization~\citep{biros2005pln1,biros2005pln2,akcelik2006parallel}.
See \citet{benzi2005nss} for a review of methods for saddle point problems.

As a practical and architectural concern, we warn that that excessive splitting leads to lower arithmetic intensity and more synchronization points which reduces the utilization of modern hardware.
Therefore, it usually only makes sense to split when there is a clear reward in superior algorithmic performance such as many fewer Krylov iterations or substantially reduced storage requirements (e.g. by not needing to assemble inter-field coupling or by taking advantage of symmetric block storage for part of a larger problem).

\subsection{Linear algebraic interfaces to facilitate field-split preconditioning}\label{sec:multiphysics:algebraic}
One ideal governing the {\PETSc} approach is to isolate the user's specification of the discretization and physics from the solution methods and underlying data structures so that solver choices can be delayed until run time.
The user should not need to write any special code to support a given solver (in some cases auxiliary matrices may still be needed), thus new solvers will automatically be available when added to the library or present in a plugin.
The seemingly simple matter of interlacing fields for high throughput (see \secref{sec:throughput}) versus splitting them for certain preconditioners motivates an interface in which ``blocks'' are addressed more abstractly than by number.
Instead of addressing a field by number, {\PETSc} uses the \code{IS} class which represents an arbitrary index set.
An \code{IS} has an MPI communicator and indices which may be represented more compactly than arrays if they have structure such as a regular stride.

There are two classes of matrix storage format relevant to field-split versus monolithic preconditioners.
The traditional compressed sparse row formats, known as \code{AIJ} (plus symmetric and node-blocked variants), are fully assembled and stored in row-partitioned form.
The \code{Nest} format does not store entries directly, instead it stores nested matrices of any format with the action of the whole matrix defined in terms of its nested blocks.
Each nested block can use a different format, enabling optimization for symmetry and constant block size, even for mixed discretizations or multi-domain problems where it would otherwise not be possible.

\cfunc|MatGetSubMatrix| is the primary interface used by \cverb|PCFieldSplit| to extract blocks from the Jacobian.
This function extracts a parallel submatrix with distribution specified by the distribution of the row and column index sets.
For monolithic matrix formats like \code{AIJ}, \cfunc|MatGetSubMatrix| necessarily requires the matrix entries to be copied into a new data structure that as much as doubles the memory needed by an application.
In the generic parallel setting, this operation requires complex communication patterns that are not memory scalable as currently implemented.
For matrices in the \code{Nest} format, \cfunc|MatGetSubMatrix| returns the submatrix without making a copy or any parallel communication.

While Schur complements needed by block factorization could be constructed from submatrices, it would be cumbersome for the user to provide problem-specific approximations to the Schur complement this way.
Instead, \code{PCFieldSplit} uses
\begin{minted}{c}
PetscErrorCode MatGetSchurComplement(Mat mat,
                   IS isrow0,IS iscol0,IS isrow1,IS iscol1,
                   MatReuse mreuse,Mat *newmat,MatReuse preuse,Mat *newpmat);
\end{minted}
to extract the Schur complement of $(\cverb|isrow0|,\cverb|iscol0|)$ restricted to $(\cverb|isrow1|,\cverb|iscol1|)$ and an approximation to be used for preconditioning.
The default implementation returns a matrix of type \cverb|SchurComplement| which applies $S$ according to its definition $D - CA^{-1}B$ by storing the four blocks, and if requested by the caller, the SIMPLE~\citep{patankar1972cph} approximation $D - C\text{diag}(A)^{-1}B$.
This function can be overridden by the user to return arbitrary problem-specific data.

The \cfunc|MatGetSubMatrix| and \cfunc|MatGetSchurComplement| interfaces are perfectly adequate for use by \code{PCFieldSplit}, and \cverb|Nest| matrices provide an efficient storage format, but since the assembly interface is different for creating separate blocks from a monolithic matrix, it seems to require the user to decide up-front whether to assemble a \cverb|Nest| or a \cverb|AIJ| matrix.
Additionally, if \cverb|AIJ| is used, assembly for a single physics would need to ``know'' about the other fields in order to set the indices correctly for insertion.
These are both highly undesirable because they limit the algorithmic choices that can be made at run time.
To overcome this unsatisfactory performance and interface complexity, I introduced a new interface for modular and generic assembly.

Before describing the interface, we need to define ``local'' spaces in the multi-physics context.
We first assume a non-overlapping partition of owned nodes across subdomains, where typically subdomains are identified with MPI processes.
We also assume a possibly overlapping partition of the integration domain.
In the continuous finite element context, assembly is done by integration over elements, with each element typically integrated on exactly one process (thus the partition is non-overlapping).
Note that finite difference and finite volume methods can also be interpreted as integration with appropriate quadratures.
The local space is then defined as the set of all nodes with support on the integration subdomain.
This is also exactly the domain on which state variables need to be defined to evaluate a subdomain's contribution to the residual and contains the residual contribution from any given subdomain (identical for continuous finite element methods).
The local space comes with an independent ordering (starting at zero for each subdomain) and a local-to-global mapping that translates local indices to global indices.
Matrices and vectors typically carry a local-to-global mapping so that entries can be referenced by local index.
A sub-physics local space is defined as the subset of a multi-physics local space that contributes to a particular physics (or field or other ``split'') and is uniquely represented by an index set (\cverb|IS|) containing those multi-physics local indices appearing in the sub-physics local space.
When \cverb|DMComposite| is used to manage a multi-physics problem, each sub-physics local index set is a contiguous range.

For the purpose of matrix assembly, the local space defines the part of the global vector and matrix into which entries can be contributed by the method outlined below.
This is a restriction relative to the standard method of inserting any entry by global index, but the user can still define the local-to-global mapping to allow insertion wherever they desire.
In return for this restriction, partially assembled matrices required by non-overlapping domain decomposition methods like FETI-DP~\citep{farhat2001feti,farhat2000scalable,klawonn2006dual,klawonn2007robust,klawonn2007inexact} and the closely related BDDC~\citep{dohrmann2003psb,mandel2003cbd,li2006bddc} can be assembled with no changes to user code.

Assembly of ``sub-physics'' blocks as well as off-diagonal coupling blocks is achieved using
\begin{minted}{c}
PetscErrorCode MatGetLocalSubMatrix(Mat A,IS rows,IS cols,Mat *submat);
\end{minted}
which returns a submatrix with local indices defined by the index sets.
This function is not collective and makes only weak guarantees about the functionality implemented by the returned submatrix.
In particular, the communicator is not specified and collective operations like {\MatMult} may not be defined.
If the matrix storage format is \cverb|Nest|, this function simply returns the appropriate submatrix, which usually lives on a parallel communicator and has full functionality.
For matrix formats that do not implement \cfunc|MatGetLocalSubMatrix|, a proxy matrix is set up on \cverb|PETSC_COMM_SELF| that implements \cfunc|MatSetValuesLocal| and similar functions by translating sub-physics local indices to coupled local or global (best choice depending on what is explicitly supported) indices and setting values in the parent matrix.
If the row and column index sets have matching block size attribute, \cfunc|MatSetValuesBlockedLocal| is also implemented regardless of whether the underlying storage format uses blocks.
This permits sub-physics modules having constant block size to always speak the most specific language (blocked and/or symmetric) regardless of the underlying format.
When the underlying format specifically supports blocks, we reap the benefits of faster insertion due to fewer searches and moves, otherwise there is negligible performance penalty.
When \cfunc|MatSetValuesBlockedLocal| is used recursively and the matrix implementation has no specific support, the proxy matrices are flattened so that index translation is never done more than once.
This interface allows separate physics modules to be packaged independently and gives a run-time choice of monolithic matrix formats or \cverb|Nest| with the best possible format in each block and high performance in all cases.


\subsection{High throughput on cache-based architectures}\label{ssec:throughput}
\begin{quote}
  \emph{The easiest way to make software scalable is to make it sequentially inefficient.}~\cite{gropp1999exploiting}
\end{quote}

\subsection{Sparse matrix kernels}\label{ssec:sparsekernels}

This section briefly describes some optimizations that improved PETSc's matrix kernels by about 30\% on Intel and AMD architectures, bringing them quite close to the memory bandwidth limits.
This was implemented subsequent to the new data structures for factored matrices described in \cite{smith2010sparse} and thus represents an additional improvement.

The following hardware descriptions generally hold across Intel Core, Core 2, and Core i7 as well as AMD K8 and K10 microarchitectures.
Most variation is due to different prefetch and TLB semantics as well as different latencies for vector instructions.
Details can be found in the optimization reference manuals~\cite{intel2011optimization,amd2009optimization} as well as Agner Fog's excellent resources~\cite{fog2011michoarchitecture,fog2011instruction}.
See \cite{drepper2007memory} for a more accessible introduction to the memory hierarchy.

Most of our optimizations involve maximal reuse of level 1 data (L1D) cache, higher level caches, and streaming bandwidth.
Each floating point unit can issue one packed add and one packed multiply per clock cycle, with latencies of 3 to 5 cycles.
SSE instructions may have both operands in registers or have one operand in L1D with no throughput penalty, although a concurrently issued add and multiply cannot both have a memory operand.
AMD has a 2-cycle latency penalty for memory operands, Intel has no latency penalty.
This means that L1D is almost as good as a register, compared to L2 with roughly 10-cycle latency and DRAM with approximately 250-cycle latency.

Assuming a 2.5 GHz clock, one packed load and store per cycle means a bandwidth to L1 of 24 GB/s each way (48 GB/s each way for newer architectures supporting 32-byte AVX registers).
We can compare this to per-core memory bandwidth ranging from 1.75 GB/s (6-core with 2-channel DDR2-667, found on Cray XT-5) to 9.6 GB/s (4-core with 3-channel DDR3-1600, found on Intel's newest ``Sandy Bridge'' processors).
With four (SSE2) or eight (AVX) double precision floating point operations possible per cycle, it becomes clear that memory will be the bottleneck unless many floating point operations can be performed per value loaded.
The ratio of flops per byte for a computational kernel is known as arithmetic intensity.
The systems above balance computation with memory when the arithmetic intensity is between 2 and 6 flops/byte.

The hardware prefetch unit can recognize 16 forward-moving streams and 4 backward-moving streams, but only one stream per 4 kB page and not across page boundaries.
Additionally, the hardware prefetcher does not resolve minor TLB misses which is a significant issue because the TLB can only address 1 or 2 MB and TLB misses produce roughly 60 cycles of latency, plus DRAM latency if the resulting address is not in high-level cache.
Software prefetch can hide these latencies and resolve TLB misses in advance, providing roughly 20\% improvement in our STREAM benchmarks~\cite{mccalpin2007stream}.
In addition to improving overall bandwidth, software prefetch instructions can set a ``non-temporal access'' (NTA) policy such that lines evicted from L1D will not subsequently reside in higher level caches.

We consider four matrix formats with increasing amounts of structure: {\AIJ}, {\AIJInode}, {\BAIJ}, and {\SBAIJ}.
{\AIJ} is a standard compressed row format that stores a column index for each nonzero entry.
A matrix-vector product $y \gets A x$ is characterized by the kernel
\begin{minted}{c}
  for (i=0; i<m; i++) {
      y[i] = 0.0;
      for (j=ai[i]; j<ai[i+1]; j++)
          y[i] += aa[j] * x[aj[j]];
  }
\end{minted}
Assuming perfect cache reuse with $m$ rows and an average of $n$ nonzeros per row, {\MatMult} has an arithmetic intensity of
\begin{align*}
  \I_{\text{AIJ}} = \frac{2n}{(n+1)\texttt{sizeof(Scalar)} + (n+1)\texttt{sizeof(Int)}} \xrightarrow{n\to\infty} 0.167 \text{flops/byte}
\end{align*}
where the long-row limit is calculated for \cverb|double| precision and 32 bit integers.
In practice, the vector will not reuse cache perfectly so some entries of $x$ will need to be loaded into cache more than once.
Choosing a low-bandwidth ordering such as Reverse Cuthill-McKee~\cite{cuthill1969reducing} (RCM) improves cache reuse which reduces end-to-end run time substantially (speeding up matrix-free operations such as residual evaluation in addition to sparse matrix kernels), e.g. by more than a factor of 2~\cite{gropp2000pmt}.
Our optimization for {\AIJ} consisted of inserting software prefetch instructions to initiate loads of every cache line holding entries in the row after the current one.
The L1D cache line length is detected during configuration so only one prefetch instruction per line is issued (it is 64 bytes on the x86-64 microarchitectures considered here).
Sparse matrix-vector products for PDEs can be thought of as a stencil operation combined with a high-volume stream of matrix entries and column indices (the weights and shape of the stencil).
Without the NTA policy, the high-volume stream flushes the vector $x$ out of caches prematurely.

The {\AIJInode} format simply augments the {\AIJ} format by marking clusters of consecutive rows ``Inodes'' that have the same nonzero pattern, then matrix kernels unroll over the Inodes.
Use of the {\AIJInode} format is automatic if the matrix has consecutive rows with the same nonzero pattern.
The column indices are \emph{not} copied out of the standard {\AIJ} storage so matrix kernels skip over $b-1$ rows of identical entries in each Inode of size $b$.
Skipping over these rows of column indices is especially inefficient with hardware prefetch because entries are eagerly brought into cache and then skipped, followed by a full DRAM stall as the kernel jumps over many skipped ahead in \cverb|aj| and starts $b-1$ new streams from \cverb|aa| that are not predicted by the prefetch unit.
Software prefetch for {\AIJInode} avoids bringing in the redundant column entries and prevents the stalls due to jumping over column indices and starting new \cverb|aa| streams.
The prefetch logic assumes that the next Inode and row lengths will be approximately the same size as the current one since this is the typical case for PDE-like problems.
If no redundant column entries are brought in, the arithmetic intensity is
\begin{align*}
  \I_{{\AIJInode}} = \frac{2nb}{(n+1)b\texttt{sizeof(Scalar)} + (n+2)\texttt{sizeof(Int)}} \xrightarrow[b=3]{n\to\infty} 0.214 \text{flops/byte}
\end{align*}
where the example Inode size of 3 is representative of 3D elasticity or the viscous part of a Stokes problem.
We expect this performance model to be somewhat less sharp than for {\AIJ} because it is not possible to load less than a cache line, therefore redundant column indices are unavoidable unless all rows naturally align with cache line boundaries.

The {\BAIJ} format optimizes for constant block size by storing only one column index per $b\times b$ block.
With this format, the problems of skipping ahead in \cverb|aj| and \cverb|aa| go away, but issues of peak bandwidth and enabling the vector to reuse cache remain.
The arithmetic intensity
\begin{align*}
  \I_{{\BAIJ}} = \frac{2nb}{(n+1)b\texttt{sizeof(Scalar)} + (n/b+1)\texttt{sizeof(Int)}} \to \xrightarrow[b=3]{n\to\infty} 0.237 \text{flops/byte}
\end{align*}
is very close to the dense limit of $0.25$.
In practice, {\BAIJ} achieves a higher fraction of the bandwidth peak due its more regular memory access and unrolled kernels.

The symmetric block format {\SBAIJ} which stores only the upper-triangular part in compressed row format has a more involved memory access pattern.
Each matrix entry is only loaded once where it is used for both upper and lower triangular contributions.
The contribution from the lower-triangular part has multiple destinations effectively doubling the number of stencil-like streams that must be sustained, with all the new streams read-write as opposed to the read-only streams of nonsymmetric row formats.
In return, the storage and arithmetic intensity is almost double that of nonsymmetric formats
\begin{align*}
  \I_{\SBAIJ} = \frac{4nb - 2b^2}{(n+2)b\texttt{sizeof(Scalar)} + (n/b+1)\texttt{sizeof(Int)}} \to \xrightarrow[b=3]{n\to\infty} 0.47 \text{flops/byte} .
\end{align*}
% let scalar = 8; int = 4; intenBAIJ n b = 2*n*b / ((n+1)*b*scalar + (n/b+1)*int); intenAIJInode n b = 2*n*b/((n+1)*b*scalar + (n+2)*int)

The data structure change described in \cite{smith2010sparse} caused both forward and back solves to traverse matrix entries forward in memory when using (incomplete) LU.
Even though the matrix entries are fetched moving forward, the stencil for ``back solve'' still moves backward through memory so there are much fewer hardware prefetch streams (4 instead of 16 forward-moving on Intel/AMD, there is no hardware prefetch for backward-moving streams on Blue Gene/P).

For (incomplete) Cholesky, the factors are only stored once so it is not possible for both forward and back solves to traverse matrix entries by moving forward.
Because the primary backward-moving streams are prefetched by software, this causes relatively little performance degradation compared to ILU.

% Core 2 P8700: DDR2-800 (6400 MB/s), single thread Triad peak 5400 MB/s
% gcc-4.7 -O3 -std=c99 -march=native BasicVersion-onnode.c -lm -lnuma -DSSE2 -DPREFETCH_NTA
% I observed as high as 5500 MB/s when using -DFAULT_TOGETHER, but this
% is not practical
% Performance is dependent on how the memory was used the last time
% through, therefore optimization of each STREAM kernel is not entirely
% independent of the others.

As an estimate of the peak usable bandwidth, we use the STREAM Triad benchmark with a moderately tuned implementation (software prefetch, SSE2 arithmetic, non-temporal stores, unrolled over cache lines).
We consider two test systems, a Core 2 Duo (P8700) clocked at 2.53 GHz with a single DDR2-800 memory channel (theoretical bandwidth of 6.4 GB/s) and a two-socket Opteron 2356 (quad core) clocked at 2.3 GHz with two channels of DDR2-677 memory per socket (theoretical peak of 5.3 GB/s per channel).
The achieved bandwidth for Triad on the Intel system was 5.4 GB/s for a single thread.
On Opteron, the single-threaded Triad achieved 5.8 GB/s, increasing to \todo{XXX} GB/s for 4 processes (2 per socket) and \todo{XXX} GB/s for 8 processes.

Table~\ref{tab:throughput:baij} shows the performance of the three primary sparse matrix kernels with each storage format on both test machines.
With prefetch enabled, the sparse matrix kernels obtain a remarkably high fraction of theoretical peak bandwidth, especially on Intel processors.
Indeed, the scalar format performance is competitive with the best {\MatMult} results of \cite{williams2007osm} and block formats are consistently faster by a factor of 20 to 30 percent.
Block formats provide greater benefit for {\MatSolve} than for {\MatMult}, with {\BAIJ} {\MatSolve} frequently outperforming the corresponding {\MatMult}.
We are not aware of previous reports of this effect and speculate that it is due to {\MatSolve} having fewer read streams and using only one vector at a time ({\MatMult} reads from one vector and writes to another, {\MatSolve} is in-place).

\begin{table}
  \centering
  \begin{tabular}{l c c c c}
    \toprule
                                                   & \multicolumn{4}{c}{Core 2, 1 process}            \\
                                                   & {\AIJ}       & {\AIJInode} & {\BAIJ}      & {\SBAIJ}     \\ \cmidrule{2-5}
    \MatMult                                       & 916(103\%) & 855(81\%) & 1013(86\%) & 1429(62\%) \\
    \MatSolve/ILU                                  & 799        & 742       & 1078       & ---        \\
    \MatSolve/ICC                                  & 741        & 737       & 1011       & 1007       \\ \midrule
                                                   & \multicolumn{4}{c}{Opteron, 1 process}           \\
                                                   & {\AIJ}       & {\AIJInode} & {\BAIJ}      & {\SBAIJ}     \\ \cmidrule{2-5}
    \MatMult                                       & 506(53\%)  & 592(52\%) & 735(58\%)  & 744(30\%)  \\
    \MatSolve/ILU                                  & 460        & 572       & 845        & ---        \\
    \MatSolve/ICC                                  & 426        & 427       & 815        & 814        \\ \midrule
                                                   & \multicolumn{4}{c}{Opteron, 4 processes}         \\
                                                   & {\AIJ}       & {\AIJInode} & {\BAIJ}      & {\SBAIJ}     \\ \cmidrule{2-5}
%% FIXME: STREAM performance using MPI
    \MatMult                                       & 1673       & 1978      & 2450       & 3089       \\
    \MatSolve/ILU                                  & 1621       & 1896      & 2766       & ---        \\
    \MatSolve/ICC                                  & 1519       & 1522      & 2700       & 2710       \\ \midrule
                                                   & \multicolumn{4}{c}{Opteron, 8 processes}         \\
                                                   & {\AIJ}       & {\AIJInode} & {\BAIJ}      & {\SBAIJ}     \\ \cmidrule{2-5}
%% FIXME: STREAM performance using MPI
    \MatMult                                       & 2374       & 2408      & 2897       & 4331       \\
    \MatSolve/ILU                                  & 2321       & 1954      & 2913       & ---        \\
    \MatSolve/ICC                                  & 2214       & 1854      & 2892       & 2877       \\
    \bottomrule
  \end{tabular}
  \caption{Throughput (Mflop/s) and percentage of STREAM Triad bandwidth, assuming optimal vector reuse, for three matrix kernels and different matrix formats.
    The test machines are a Core 2 Duo (P8700) and an Opteron 2356 (two sockets).
    The test problem is a $Q_1$ finite element discretization with block size of 2 on a 3D mesh (essentially a 27-point stencil) in which each process has $64\times 64\times 63$ nodes.}\label{tab:throughput:baij}
\end{table}

\subsection{Small dense tensor product kernels}\label{ssec:tensor}
A fundamental operation for computing with high order finite element and spectral element methods is application of a 3D tensor product kernel operations $y \gets (A\otimes B\otimes C) x$, or, in index notation,
\begin{equation}\label{eq:tensor:kernel}
  y_{ijk} \gets A_{i\alpha} B_{b\beta} C_{c\gamma} x_{\alpha\beta\gamma} .
\end{equation}
To evaluate a state vector at quadrature points, the Greek indices run over the basis functions in each Cartesian direction and the Latin indices run over the number of quadrature points in that direction.
When using a Gauss quadrature that integrates the mass matrix exactly (a typical choice in finite element computations), the range is the same.
Additionally, when the approximation order is isotropic (the most common case), all the ranges are the same.
The tensor product operation can be written in various ways as dense matrix multiplication (albeit highly non-square) so we expect it to achieve very close to the floating point peak for sufficiently large sizes (e.g. by calling a tuned BLAS3).
Increasing the number of fields also increases the problem size which improves the efficiency of BLAS3 calls.
The attainable performance for smaller sizes is much less clear, hence we examine the scalar $\Qk 3$ case.
This is a very practical approximation order: it is high enough to reap some benefits from the regular structure, but not so high as to make resolving geometry overly difficult or to have highly non-uniform resolution (due to clustering of interpolation nodes near the edges of elements as the approximation order increases).

In the scalar $\Qk 3$ case, each of $A,B,C$ is $4\times 4$ and the entire operation produces 64 entries in $y$ from 64 entries in $x$.
We have considered a variety of unroll-and-jam optimizations as well as different register blocking strategies.
The SSE3 horizontal add instruction \asm{HADDPD} has increased latency and reduced throughput compared to the standard vector addition and multiplication instructions \asm{ADDPD} and \asm{MULPD}, therefore we should try to avoid it whenever possible.
Assuming lexicographic ordering, applying the first two parts $A$ and $B$ of the tensor product kernel perform the same operation for all values of $\gamma$ and thus provide balanced adds and multiplies (which can be issued concurrently as long as the kernel has been unrolled enough to cover the latency) without needing any horizontal operations.
This is not possible when applying $C$ because packed loads (and vector arithmetic with memory operands) retrieve consecutive entries.
The input could be transposed to avoid this constraint, but the cost for small sizes overwhelms the benefit, therefore we are forced to use one \asm{HADDPD} instruction per packed result $y_{i,j,k:k+1}$ and the reduction also creates some unavoidable (without more registers) pipeline stalls.
Application of $A\otimes B\otimes \bm 1$ achieve nearly 80\% of floating point peak on the Core 2 Duo while $C$ is reduced to near 60\%.
The full tensor product contraction $A\otimes B\otimes C$ achieves more than 70\% of floating point peak on Intel and about 60\% on AMD.
This performance is comparable to or better than the best autotuned results of \cite{shin2010speeding}, though they achieve slightly higher performance for larger sizes (which have less overhead).
Note that large size tensor contractions are important for quantum chemistry and have thus received more optimization attention, \eg \cite{kaushik2008improving,hirata2003tensor}.
Contrast the tensor product performance with sparse matrix kernels that attain less than 10\% of floating point peak due to low arithmetic intensity.

An old \cite{mccalpin2007stream} but continuing trend in computer architecture is for peak floating point performance to continue to improve via the use of longer registers (Intel's AVX and the Blue Gene/Q architecture have packed 32-byte floating point registers), more cores, and GPU-style vectorization while memory bandwidth increases much more slowly~\cite{keyes2011exaflop}.
An additional feature of upcoming high-performance computing architectures is the prevalence of in-order execution; high power requirements no longer justify the luxury of sophisticated out-of-order execution units common on today's commodity hardware~\cite{seiler2008larrabee,pham2006overview}.
Achieving high performance on such systems requires decomposing the computation into kernels with high arithmetic intensity (so as not to overload the memory subsystem) and sufficient local structure to utilize vector registers and effectively schedule instructions at compile time.
We optimized streaming stencil kernels for Blue Gene/P in \cite{malas2011streaming}, including implementation of a static scheduler that we used to rapidly perform loop optimizations with a small assembly-language building block that exploited the available SIMD floating point unit.
With this approach, we were able to effectively use all the floating point and general purpose registers to hide instruction latency, delivering 93\% of theoretical FPU peak for the 27-point stencil with problem sizes that fit in level 1 cache and 91\% of bandwidth for larger problem sizes (72\% of FPU peak).
These results are 80\% better than the best previously published for Blue Gene/P.

Since the tensor product kernel has sufficient local structure to utilize vector registers, we can estimate its performance on future architectures by computing its arithmetic intensity.
For simplicity, suppose that $n = p+1$ is both the number of interpolation nodes and quadrature points on an element in each Cartesian direction, and that $b$ is the number of degrees of freedom per node.
Then direction of a contraction performs $bn^4$ multiplies and $b(n-1)n^3$ additions on the reference element.
We approximate the total across all three dimensions $3bn^4$ fused multiply-add (FMA) operations because FMA units are increasingly popular.
When evaluating both interpolation and gradient (most general form), it is possible to share partial results so that both can be evaluated in $9bn^4$ FMAs.
Translating the gradient from reference to physical element costs another $9bn^3$ FMAs.
The Galerkin procedure evaluates these contractions twice per residual or matrix-free Jacobian (once to evaluate the trial function, once more in transpose for the test functions), leading to a total of $18(bn^4 + bn^3) + Qn^3$ FMAs, where $Q$ is the number of operations per quadrature node.
Typically we do not store the coordinate transformation, so it must also be computed at a cost of $9\cdot 3n^4 + 31n^3$ (assume block size 3 and held in an equal order basis, evaluate only on the reference element, then invert the local Jacobian at every quadrature node at a cost of 30 FMA or multiplies and one division).
Thus, for a problem with $b$ degrees of freedom per node and no sharing between elements, we have total operation count of $9(3 + 2b)n^4 + (31+18b + Q)n^3$ with $(b+3)n^3$ floating point values coming in and $bn^3$ out.

To understand these tradeoffs, we consider three block sizes at different approximation orders and plot the memory bandwidth and number of flops (calculated as twice the number of FMAs) required to compute each entry in the output vector.
The performance model for assembled sparse matrices assumes {\BAIJ} storage and perfect vector reuse within the cache system.
The model for unassembled storage applied by tensor product assumes a general tensor-product quadrature of equal order, that coordinates are stored in a function space of the same order as the independent variables and are needed to compute the coordinate transformation but do not contribute to a result, that the physics can be represented using stored data equivalent in size to a stored gradient plus stored function values, and that the physics can be applied using an operation similar in structure to Newton-linearization of Navier-Stokes with power-law rheology plus a reaction term coupling all components.
This is intended to be nearly a worst-case setting for the unassembled tensor product formulation; it is cheaper for simpler physics, simpler coefficients, collocated quadratures, etc.

\begin{figure}
  \centering
  \includegraphics{TensorVsAssembly}
  % python2 ./spmvmodel.py --plot --format pdf -o TensorVsAssembly.pdf
  \caption{Memory and floating point requirements for matrix-free tensor-product application of an operator versus representation as an assembled matrix stored in \BAIJ[b] format.
    The same operation $y \gets A x$ is applied in both cases, the storage is just different.
    A ``result'' is a single scalar entry in $y$, regardless of the block size $b$.}\label{fig:tensorasm}
\end{figure}

The memory and floating point costs are shown in \figref{fig:tensorasm}.
The unassembled representation is uniformly better in terms of memory use for all orders $p \ge 2$, but has significant floating point overhead for the smallest sizes, especially with small block size.
Note that both bandwidth and computation requirements improve for the tensor product formulation when the block size increases (because the overhead of the coordinate transformation is reduced), while they degrade for the assembled representation.
The arithmetic intensity for unassembled representations is typically around 8 flops/byte with weak dependence on polynomial order, block size, and even details of the physics (unless a great deal of recomputation is required for linearization, or a nonsymmetric high-rank tensor needs to be stored).
This should be compared to the assembled sparse matrix representation which cannot surpass 1 flop/4 bytes even if column indices were not stored.
As noted in \secref{ssec:sparsekernels}, modern hardware balances computation with memory when the arithmetic intensity is between about 2 and 6 flops/byte and expected to increase.
The 30-fold increase in arithmetic intensity while simultaneously reducing memory traffic below even the lowest order representation bodes well for the relevance of unassembled Jacobian representations on future architectures.


\section{Discretization minutia}\label{sec:discretization}
\subsection{Regularity and boundary conditions}\label{ssec:slip}
\subsection{Dirichlet boundary conditions}\label{ssec:implementation-dirichlet}
In the continuum context, Dirichlet boundary conditions are built into the approximation space and thus do not explicitly appear in the weak form.
There are many ways to implement Dirichlet boundary conditions in the discrete context including removal from the ansatz space, penalties, ``lifting'' the known part to the right hand side for the linear problem, zeroing rows of the Jacobian, and zeroing both rows and columns by suitable evaluation of the residual.
Most of these methods perform similarly for simple problems and simple preconditioners, but have serious deficiencies for more difficult problems and sophisticated solvers.
When possible (e.g. the Dirichlet part of the domain is not itself part of the solution such as occurs with a frozen bed or contact problem), removal of Dirichlet unknowns is a robust solution, but it introduces some complexity in managing vectors residing in the ansatz space versus vectors residing in the closure (used for output and visualization) and prevents direct addressing of neighbors in structured grid computation.
Removal of Dirichlet degrees of freedom is the standard way to enforce Dirichlet conditions in \Dohp, the function space provides access to the state as part of a homogeneous or inhomogeneous space.
When removal is not practical, other approaches requiring user involvement must be used.

Penalties are easy to implement, but a penalty parameter must be chosen which contributes to ill-conditioning which reduces the accuracy of the solution (ability to converge to very high tolerance), can contaminate Schur complements, requires the Krylov method to work in the preconditioned norm (usually means left preconditioning instead of right), and must always be paired with a preconditioner that corrects the contribution from the penalty.
``Lifting'' can be performed directly on the linear system and is performed transparently using \PETSc's \code{PCRedistribute}, but the method does not compose well with field-split and hierarchical preconditioners, and the sparse matrix manipulations require extra memory and time.
Simply replacing rows of the Jacobian matrix with rows of the identity is easy to perform independent of the element assembly loop.
Unfortunately, it destroys symmetry and pollutes the spectrum of the operator which has very problem-dependent effects on the iteration count.

Zeroing both the rows and columns corresponding to Dirichlet degrees of freedom is the best alternative when the degrees of freedom are not eliminated, but there are several implementation details to consider.
Assembling into the matrix without observing boundary conditions and zeroing afterward is not efficient with sparse matrix representations since columns are difficult to address in a compressed row format (and rows are difficult to address for compressed column formats).
A better approach is to discard contributions to those rows and columns during insertion of the element stiffness matrix and then simply place the diagonal entry afterward.
This requires a compatible residual evaluation which we now consider.

Let $\VV_D$ be a discrete ansatz space with inhomogeneous Dirichlet boundary conditions implicitly built in, $\VV_0$ be the corresponding space with homogeneous conditions, $\VV_\Gamma$ be the trace space on the Dirichlet boundary, and $\bar\VV = \VV_0 \times \VV_\Gamma$ the space of all functions in the finite element space defined on the closure of the domain.
We define three projectors on $\bar\VV$,  $R_0$ projects to the $\VV_0$ subspace of $\bar\VV$ with zero values on the boundary $\VV_\Gamma$, $R_D$ projects to the affine subspace $\VV_D$, and $R_\Gamma$ projects to the trace space $\VV_\Gamma$ with zero in the interior.
We can now write the discrete residual $F$ in terms of the ``interior'' residual $f$ that does not recognize boundary conditions as
\begin{equation}\label{eq:slip:dirichlet}
  F(u) = R_0 f(R_D u) + \alpha R_\Gamma (u - R_D 0) .
\end{equation}
The value of the possibly spatially varying scaling factor $\alpha \ne 0$ does not affect the correctness of the formulation, but weighting it to be of similar magnitude to nearby diagonal entries in the matrix (e.g. by using local viscosity and the mesh size) is preferred to improve the conditioning of the linear system including boundary conditions and more importantly when geometric multigrid is used with rediscretized (non-Galerkin) coarse level operators.
The Jacobian of \eqref{eq:slip:dirichlet} isolates the Dirichlet degrees of freedom from the rest of the system.
The implementation of \eqref{eq:slip:dirichlet} and its derivative is straightforward, local element residuals and Jacobians are evaluated with correct Dirichlet values imposed on the state $u$, the result is inserted with contributions to Dirichlet nodes discarded.
Then a loop over the boundary places the $\alpha$-scaled difference from the correct boundary values into the residual vector and inserts $\alpha$ on the diagonal of the Jacobian.

\subsection{Slip}
Slip boundary conditions are a combination of Dirichlet on the normal component and (usually nonlinear) Robin on the tangent component.
When the slip surface is curved, there are multiple ways to define the normal direction.
Geometric averages such as those advocated by \citet{walkley2004calculation} can be very accurate and appear to be preferable for problems with surface tension in which conservation is not essential.
When exact conservation is critical, there is no choice but to use ``conservative normals''~\citep{lynch1980finite}.
Conservative normals are defined at an arbitrary node (or mode) $i$ in which the basis function $\phi_i$ has support on the boundary $\Gamma$ by
\begin{equation}
  \label{eq:slip:conservative-normal}
  \nn_i = \frac{\int_\Gamma \phi_i \nn}{\abs{\int_\Gamma \phi_i\nn}} .
\end{equation}
If the velocity field is constrained so that $\uu_i \cdot \nn_i = 0$ for a node $i$ with support on the boundary, then node $i$ will contribute zero flux through the boundary.
We enforce this condition for all boundary nodes so the total flux is also zero across the boundary: $\int_\Gamma \uu \cdot \nn = 0$.
Note that we do not in general have that $\int_f \uu \cdot \nn = 0$ for all mesh faces $f$ for the same reason that continuous Galerkin methods are not locally conservative.
However, a weaker local conservation statement similar to that in \citet{hughes2000continuous} still holds.

There is a technical difficulty observed by \citet{walkley2004calculation} when using conservative normals with inf-sup stable spaces.
In particular, the velocity space must be at least quadratic for stability reasons and the corner basis functions of $\Pk{2}$ triangles (appearing on the surface of a tetrahedral mesh) have the property that
\begin{equation*}
  \int_\Gamma \phi_i = 0 .
\end{equation*}
This means that the normal need not be constrained on any flat element face (but having no constraint admits non-physical recirculation within elements) and that for isoparametrically mapped elements with small face curvature, the definition of the normal becomes unstable due to near cancellation and can flip direction as the surface evolves.
This problem does not occur for non-deformed quadrilaterals (on the surface of a hexahedral mesh), but may arise for sufficiently deformed elements.
The extra constraint on element quality is inconvenient for moving mesh simulations and for meshing complex structures, but it is a limitation that we accept in the present work.
An interesting alternative is to eschew Lagrange interpolants in favor of non-negative bases such as Bernstein polynomials or more generally, the spline bases used in isogeometric analysis~\citep{cottrell2009isogeometric}; see \citet{akkerman2011isogeometric} for recent results with free surface flows.

When implementing the normal constraint in slip boundary conditions, it is not practical to remove the normal component from the ansatz space so we prefer to leave them in as described in \secref{ssec:implementation-dirichlet}.
To enforce Dirichlet conditions on the normal component, we rotate coordinates in the global vector so that the normal component is isolated.
The rotation is undone at the element level so that local operations need not be aware that the solution vector contains rotated blocks.
This rotation affects the construction of coarse level spaces in multigrid and domain decomposition methods.
The ML~\citep{ml-guide} algebraic multigrid package allows the user to specify low-energy modes to be represented in the coarse space.
These modes should respect the change of basis so that their energies remain low.
The impact on domain decomposition methods such as FETI-DP is more delicate, see \citet{klawonn2007robust,klawonn2006dual,dohrmann2010hybrid} for details on the construction of coarse spaces.

While use of conservative normals provides exact conservation across the interface, there is no guarantee that it will preserve realistic steady states, particularly for free surface flows where the hydrostatic contribution to pressure cannot be removed.
Indeed, with a smooth curved boundary and a level surface, the momentum residual will not point in the same direction as the conservative normal.
The tangent component of the momentum residual is a spurious tangent force that causes non-physical recirculation within the fluid domain that can not be prevented simply by adding artificial friction to the slip surface.
This issue is addressed \citet{behr2004application} which we follow below.
Recall the boundary integral appearing in the weak form of the Stokes problem \eqref{eq:slip:stokes-weak}
\begin{equation}\label{eq:slip:stress-bc}
  - \int_\Gamma \vv\cdot (\eta D\uu - p\bm 1)\cdot\nn
\end{equation}
where well-posedness of the continuous weak form requires specifying the stress $(D\uu - p\bm 1)\cdot\nn$ as an algebraic function of $\uu$.
We are only concerned with the tangential part of the stress since the normal components have Dirichlet conditions imposed.
The approach of \citet{behr2004application} integrates \eqref{eq:slip:stress-bc} ``as is'', without applying any boundary condition at all.
This has the effect of extending the PDE to include the boundary.
It is completely invalid for the continuum problem, resulting in non-uniqueness since the solution to the same PDE on any extended domain with any applied boundary conditions is also a solution on the initial domain, but turns out to be valid in the discrete context.
The idea was introduced in the context of open boundary conditions for natural convection in \citet{papanastasiou1992nob} and later refined 
in the restricted context of outflow boundary conditions for advection-diffusion by \citet{griffiths1997nbc} and \citet{renardy1997inb}.
In particular, \citet{griffiths1997nbc} showed that the boundary condition produces $\bigO((h+1/\Peclet)^{p+1})$ errors in $\Linfty$ for mesh size $h$, Peclet number $\Peclet$, and finite elements of polynomial degree $p$, a result distinctly better than the $\bigO(h^{p+1} + 1/\Peclet)$ obtained for Neumann outflow conditions.


\subsection{Boundary layer processes}\label{ssec:boundary-layer}
\input{boundary-layer}

\subsection{Solving steady-state problems}\label{ssec:steady-state}
\input{steady-state}

\newpage
\addcontentsline{toc}{section}{References}
\bibliographystyle{amsalpha}
\bibliography{jedbib/jedbib}

\appendix

\end{document}

