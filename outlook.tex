There are many periodic processes in glaciology, including ice stream margin migration and shutdown~\citep{raymond2000energy,bueler2009shallow}, Heinrich events~\citep{heinrich1988cyclic,calov2010heino}, and grounding line migration \citep{schoof2007isg,mismip}.
All numerical studies to date have explored these phenomena by direct time integration.
This involves large spin-up times and many time steps (sometimes millions) to complete a period.
An alternative is to pose the time-periodic problem as a nonlinear rootfinding problem and apply Newton-Krylov techniques.
This was done by \citep{merlis2008fast} for the spin-up of an ocean general circulation model and provided a factor of 10 to 100 speedup compared to direct time integration.
Furthermore, it provided access to stable limit cycles and equilibrium solutions that could not be accessed by direct time integration.
With such a method, it would also be possible to efficiently explore the effect of parameters on these cycles.
For example, using multi-parameter continuation methods~\citep{allgower2003inc} to find combinations of parameters that cause a qualitative change in the structure of the periodic solution.



\todo{write this}
