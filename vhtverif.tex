To verify the correctness of the implementation, we consider a manufactured solution with rich structure and choose a parameter range to activate all the terms in \eqref{eq:vhtstrong} and constitutive relations.
Instead of the physical parameters in \tabref{tab:vhtconst}, we take all parameters to be of order one, with solid and melt densities of 1 and 2 respectively.
For this problem, the melt fraction rises as high as \SI{28}{\percent} and temperature ranges \SI{11}{\percent} of its absolute value.
Since the activation volume $V$ and Clausius-Capeyron gradient $\beta_{CC}$ are relatively large, the pressure plays a significant, and occasionally dominant role in defining the temperature and the material rheology.
The large moisture content and large density contrast also increase the strength of the nonlinearity.
The chosen solution is given by
\begin{equation}\label{eq:vhtmanufactured}
  \begin{split}
    \rho u & = \frac 1 3 \sin \frac{\pi x}{2} \cdot \sin \frac{\pi y}{2} \cdot \sin \frac{\pi z}{2}  \\
    \rho v & = -\frac 1 3 \cos \frac{\pi x}{2} \cdot \cos \frac{\pi y}{2} \cdot \sin \frac{\pi z}{2} \\
    \rho w & = -\frac 2 3 \cos \frac{\pi x}{2} \cdot \sin \frac{\pi y}{2} \cdot \cos \frac{\pi z}{2} \\
    p      & = 1 + \cos \frac{\pi x}{2} \cdot \sin \frac{\pi y}{2} \cdot \sin \frac{\pi z}{2}        \\
    E & = \sin \pi \frac{x+y+z^2}{2} \cdot \cos \pi \frac{x^2+y+z}{2} .
  \end{split}
\end{equation}

Due to the non-uniform flow and various constitutive nonlinearities, nondimensional numbers are spatially variable.
The Reynolds number ranges up to about \num{2.4}, the Peclet number ranges up to \num{5.3}, and the Prandtl number ranges from \num{0.6} to 1.
A computed solution, accurate to \SI{0.5}{\percent} is shown in \figref{fig:vhtexact}.
As usual, this manufactured solution is not physically realizable, but it excercises all the terms.

\begin{figure}
  \centering\includegraphics[width=\textwidth]{visit0020}
  \caption{The numerical approximation to the manufactured solution \eqref{eq:vhtmanufactured} as computed using a $\Qk 3 - \Qk 2 - \Qk 3$ finite element discretization on a $12\times 12\times 12$ mesh.
    The converging streamlines are symmetric from below.
    Energy isosurfaces are shown, with the phase transition occuring at approximately $E=0$.
    The numerical solution is accurate to 4 digits for momentum and energy and 2 digits for pressure, evaluated using the maximum norm and a higher order quadrature rule.
    The gradients are accurate to 3 digits for momentum and energy, with \SI{3}{\percent} error in the pressure gradient.}\label{fig:vhtexact}
\end{figure}

We consider norms for a $\Qk 3 - \Qk 2 - \Qk 3$ approximation under $h$-refinement.
Due to the direct appearance of pressure in the equations, we cannot expect to realize fourth order convergence, at least not for the energy equation.
Figure~\ref{fig:vhtrefine} shows the observed convergence behavior in which energy clearly converges with only third order accuracy.
I do not have an explanation for why the convergence for the energy equation eventually stagnates.
It could be an artifact of the manufactured solution process (perhaps rectifiable using more accurate quadrature for the forcing term), other quadrature errors due to nonlinearity, or stability of the continuum equations or discretization.

\begin{figure}
  \centering\includegraphics{vhtdisc}
  \caption{Convergence rates for $\Qk 3 - \Qk 2 - \Qk 3$ under $h$-refinement.}\label{fig:vhtrefine}
  % ./vhtconvergence.py --plot -o vhtdisc.pdf
\end{figure}

The nonlinear solver converges quadratically as seen in \tabref{tab:vhtsnes}.
All subsequent examples have exhibited similar convergence behavior.
\begin{table}
  \centering
  \begin{tabular}{lllll}
    \toprule
    Iteration & Mass         & Momentum     & Energy       & Total        \\
    \midrule
    0         & 2.142762e-01 & 1.431024e+01 & 2.742861e+01 & 3.093796e+01 \\
    1         & 1.086178e-05 & 8.386431e+00 & 8.412471e+00 & 1.187863e+01 \\
    2         & 2.928430e-06 & 4.103421e+00 & 3.579857e+00 & 5.445497e+00 \\
    3         & 1.744093e-06 & 3.059956e+00 & 6.853340e-01 & 3.135764e+00 \\
    4         & 8.688964e-07 & 1.891518e+00 & 2.980380e-01 & 1.914854e+00 \\
    5         & 4.952597e-07 & 6.852763e-01 & 7.214378e-02 & 6.890634e-01 \\
    6         & 1.430063e-07 & 4.827890e-02 & 6.381075e-03 & 4.869877e-02 \\
    7         & 1.057706e-08 & 3.257086e-05 & 3.007905e-05 & 4.433521e-05 \\
    8         & 1.759267e-11 & 4.249319e-10 & 1.387815e-10 & 4.473666e-10 \\
    \bottomrule
  \end{tabular}
  \caption{Nonlinear convergence rates.}\label{tab:vhtsnes}
\end{table}

Numerical experiments with nonlinear solver convergence and pseudo-transient continuation~\citep{coffey2003ptc,kelley1998cap} indicates that this system does not always have steady-state solutions, and when steady-state solutions exist, they may be non-unique.
It would be interesting to explore these uniqueness properties using bifurcation techniques such as those in \citet{allgower2003inc}.
However, these phenomena have not been observed in parameter ranges that are realistic for ice flow, so we do not explore them further.
Note that such nonlinear effects do occur in glaciology, but generally involve sliding and/or geometry change.
With thermally-induced density variation, gravity, and a boundary heat source, there are not steady-states above a critical Rayleigh number.
This is seen in mantle convection and other fields, for which \eqref{eq:vhtstrong} is also valid, therefore it is no surprise that steady-states are not present.

For Dirichlet problems, pressure is only determined up to a constant.
As far as the solver is concerned, this is easy to handle by applying the Krylov iteration with null space removed and using a preconditioner that is tolerant of the one dimensional null space.
Most preconditioners other than direct solvers are suitable, and the methods of \secref{sec:multiphysics:fieldsplit} perform fine as long as inner solvers (if used) are also informed of the null space.
However, unlike for linear and power-law Stokes problems, the absolute value of pressure affects the other variables.
Thus there is a one-parameter family of solutions in which all variables change rather than only the pressure.
Additionally, it is possible to compute a negative pressure during the nonlinear solve.
This is most common at higher viscosity because the forcing terms in the manufactured solutions become huge, thus producing extreme values of pressure.
Negative pressures produce feedback through the activation volume $V$ to generate exponentially large viscosity.
