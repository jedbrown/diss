We consider the problem of computing a steady-state enthalpy (in the form of temperature and moisture) field coupled to a Stokes problem with an enthalpy-dependent rheology.
This problem is most important as part of inversion for a temperature field that is compatible with the observed geometry, flow field, borehole temperature measurements, and any other field observations.
Time-stepping a transient model to steady state is extremely expensive, therefore it will never be feasible as part of the functional to be minimized by an optimization algorithm.

Given a Lipschitz domain $\Omega$ with surface $\Gamma_s$ and time interval $(0,\tau)$, the strong form is to find total momentum, pressure, and total energy density $(\rho\uu,p,E) \in W^{1,\pfrak} \times L^2 \times H^1$ such that
\begin{subequations}\label{eq:vhtstrong}
  \begin{align}
    (\rho\uu)_t + \div (\rho\uu\otimes\uu - \eta D\uu_i + p\bm 1) - \rho \bm g &= 0 \label{eq:vhtstrong:momentum} \\
    \rho_t + \div \rho\uu &= 0 \label{eq:vhtstrong:mass} \\
    E_t + \div (E\uu_i - k_T\nabla T - L \kappa_\omega\nabla\omega) - \eta D\uu_i\tcolon D\uu_i &= 0 \label{eq:vhtstrong:energy}
  \end{align}
\end{subequations}
on $\Omega\otimes (0,\tau)$, with Dirichlet flow boundary conditions except at the free surface, and all Dirichlet boundary conditions for energy $E$.
These equations represent conservation of momentum, mass, and energy respectively.
The energy equation consists of a transport term, thermal diffusion, moisture diffusion, and heat production due to strain heating.
We switch immediately to the steady-state form of \eqref{eq:vhtstrong} in which $(\rho\uu)_t$, $\rho_t$, and $E_t$ are all zero.
Constitutive relations are needed for total density $\rho$ (\si{\kilogram\per\metre}), ice velocity $\uu_i$ (\si{\metre\per\second}), temperature $T$ (\si{\kelvin}), volumetric moisture fraction (porosity) $\omega$ (nondimensional), and viscosity $\eta$ ($\si{\pascal\second} = \si{\kilogram\per\metre\per\second}$).
In general, each constitutive relation is a function of all field variables.
The thermal conductivity $k_T$ (\si{\joule\per\metre\per\kelvin\per\second}) and hydraulic diffusivity $\kappa_\omega$ (\si{\kilogram\metre\per\second}) are taken to be constant because experimental data are sparse, but this assumption is in no way critical.

The constitutive relations for temperature and moisture fraction are usually defined piecewise.
It is preferable for the convergence of Newton methods~\citep[\cf][]{gropp2000globalized} to have a discretization with $C^1$ continuity and it is simpler for manufactured solutions if the constitutive relation has a global (\ie not piecewise) definition in terms of analytic functions.
To produce globally smooth approximations to the functions defined piecewise, we use introduce the (low precedence, infix) splice function
\begin{equation*}
  \splice{a}{b}{x_0}{\delta}{x} = a(x) + \frac{b(x)-a(x)}{2} \Big(1+\tanh \big(\frac{x-x_0}{\delta} \big) \Big)
\end{equation*}
which joins the functions $a(x)$ and $b(x)$ at $x = x_0$ via a smooth transition of characteristic width $\delta$.
Taking $\delta \to 0$ recovers the piecewise function.
This definition of $\spliceop$ is not monotone when joining two overlapping functions such as $a(x) = 0$ and $b(x) = x$ at $x_0 = 0$.
Lack of monotonicity can violate the entropy inequality, therefore a monotone spline would be preferable in real applications.

For convenience in defining constitutive relations, we introduce specific internal energy $e$ (\si{\joule\per\kilogram}) which is related to total and kinetic energy through
\begin{equation*}
  E = \rho e + \half (1-\omega) \rho_i \abs{\uu_i}^2 + \half \omega \rho_w \abs{\uu_w}^2
\end{equation*}
which we approximate as
\begin{equation}\label{eq:intenergy}
  E = \rho e +  \frac{1}{2\rho} \abs{\rho\uu}^2 .
\end{equation}
This approximation may be violated, for example, at moderate porosity when the velocity of the melt fraction is very high compared to the bulk velocity, such as in an actively draining moulin.
In such circumstances, it is likely unavoidable to add additional variables for water momentum and energy.
Consistent with exact incompressibility, the internal energy is independent of pressure even though observable quantities like temperature and moisture fraction are dependent on pressure.
Removing this assumption would produce acoustic waves and a conservative formulation would require that density be an explicit degree of freedom.
When combined with the closure for $\rho(p,e)$, shown below, \eqref{eq:intenergy} requires solving an implicit equation.
This implicit equation can always be reduced to one dimension and can be solved explicitly for some definitions of $\spliceop$.
For some purposes, it is acceptable to simply take $\rho \approx \rho_i$.
%, therefore we use the density of pure ice $\rho \approx \rho_i$ and the ice velocity $\uu_i = \uu$ (valid for $\omega \ll 1$) in this equaiton \emph{only}.
The convective contributions to momentum balance $\rho\uu\otimes\uu$ and kinetic energy $\frac{1}{2\rho}\abs{\rho\uu}^2$ have vanishing influence in glaciology, but are easy to accommodate so we keep them for completeness.
An alternative would be to discretize using $\rho$ instead of $p$ as the independent variable, but near incompressibility and the variation due to moisture content causes the resulting system to be extremely ill-conditioned.

The closures for \eqref{eq:vhtstrong} are
\begin{subequations}\label{eq:vhtclosure}
  \begin{align}
    \rho(p,e)        & = \big(1-\omega(p,e) \big) \rho_i + \omega(p,e) \rho_w                            \\
    \uu_i(\uu,p,e)   & = \uu + \rho(p,e)^{-1} \kappa_\omega \nabla\omega(p,e) \label{eq:vhtclosure:uice} \\
    T(p,e)           & = \splice{T_0 + \frac{e}{c_i}}{T_m(p)}{e_m}{\delta}{e}                            \\
    \omega(p,e)      & = \splice{0}{\frac{e-e_m(p)}{L}}{e_m}{\delta}{e}                                  \\
    \eta(\gamma,p,e) & = B(p,e)\left(\epsilon^2 + \frac{\gamma}{\gamma_0} \right)^{\frac{\mathfrak{p}-2}{2}}
  \end{align}
\end{subequations}
with second invariant $\gamma = \half D\uu\tcolon D\uu$ and the additional constitutive relations
\begin{align*}
  T_m(p)   & = T_3 - \beta_{CC} p          \\
  e_m(p)   & = c_i \big(T_m(p) - T_0 \big) \\
  T^*(p,e) & = T(p,e) - T_m(p) + T_3       \\
  B(p,E)   & = B_0 \exp \left( \frac{Q\big(T_0-T^*(p,e)\big) - pVT_0}{\nfrak R T_0 T^*(p,e)} \right) \big(1 + B_\omega \omega(p,E) \big)^{-1/\nfrak}
\end{align*}
with physical constants given in \tabref{tab:vhtconst}.
Since dimensional units are used here unlike in earlier sections, the strain rate regularization $\epsilon$ is now a fraction of $\gamma_0$ which is the second invariant of a reference strain rate.
The Arrhenius relation is normalized to zero pressure and a reference temperature $T_0$.
This formulation is not conventional in glaciology, but permits more intuitive understanding of parameters because they are no longer sensitive to the power law exponent $\pfrak$ and large exponential terms.

\begin{table}
  \centering
  \begin{tabular}{clll}
    \toprule
    Symbol          & Value                                         & Description                                                       \\
    \midrule
    $c_i$           & \SI{2009}{\joule\per\kilogram\per\kelvin}     & Specific heat capacity of ice                                     \\
    %$c_w$          & \SI{4170}{\joule\per\kilogram\per\kelvin}     & Specific heat capacity of water                                   \\
    $k_T$           & \SI{2.1}{\watt\per\metre\per\kelvin}          & Thermal conductivity of ice                                       \\
    $\rho_i$        & \SI{910}{\kilogram\per\metre\cubed}           & Density of ice                                                    \\
    $\rho_w$        & \SI{1000}{\kilogram\per\metre\cubed}          & Density of liquid water                                           \\
    $L$             & \SI{3.34e5}{\joule\per\kilogram}              & Latent heat of fusion                                             \\
    $g$             & \SI{9.81}{\metre\per\second\squared}          & Gravitational acceleration                                        \\
    $\kappa_\omega$ & \SI{1.045e-4}{\kilogram\per\metre\per\second} & Hydraulic diffusivity of ice                                      \\
    %$K_w$          & \SI{1.045e-4}{\kilogram\per\metre\per\second} & Temperature ice diffusivity
    $\beta_{CC}$    & \SI{7.9e-8}{\kelvin\per\pascal}               & Clausius-Capeyron gradient                                        \\
    $T_3$           & \SI{273.15}{\kelvin}                          & Triple point of water                                             \\
    \midrule
    $\gamma_0$      & $\half (\SI{1e-10}{\per\second})^2$           & Second invariant of reference strain rate                         \\
    $T_0$           & \SI{260}{\kelvin}                             & Reference temperature                          \\
    $B_0$           & \SI{8.56e14}{\pascal\second}                  & Viscosity at reference strain rate and temperature                \\
    $Q$             & \SI{6.0e4}{\joule\per\mole}                   & Activation energy for creep                                       \\
    $V$             & \SI{-13.e-6}{\metre\cubed\per\mole}           & Activation volume for creep                                       \\
    $R$             & \SI{8.31441}{\joule\per\mole\per\kelvin}      & Ideal gas constant                                                \\
    $B_\omega$      & 181.25                                        & Influence of water content on viscosity \citep{greve2009dynamics} \\
    \bottomrule
  \end{tabular}
  \caption{Physical constants used for the viscous heat transport problem.
    The same constants are used in \citet{aschwanden2011enthalpy}.}\label{tab:vhtconst}
\end{table}

The definitions of the moisture flux in \eqref{eq:vhtstrong:energy} and ice velocity $\uu_i$ require further explanation.
The total momentum can be defined in terms of constituent momenta as
\begin{equation}\label{eq:wmomentum}
  \begin{split}
    \rho\uu & = (1-\omega) \rho_i \uu_i + \omega \rho_w \uu_w                             \\
            & = (1-\omega) \rho_i \uu_i + \omega \rho_w u_i + \omega\rho_w(\uu_w - \uu_i) \\
            & = \rho \uu_i + \omega\rho_w(\uu_w - \uu_i) .
  \end{split}
\end{equation}
The second term is the momentum of the moisture content in the reference frame of the ice.
The mass flux of the moisture is $-\kappa_\omega\nabla\omega$ which is also the momentum density.
That is, the integral of $-\kappa_\omega\nabla\omega$ over a surface element is the mass flux (\si{\kilogram\per\second}) through that surface, while the integral over a volume element is the momentum (\si{\kilogram\metre\per\second}) of that volume.
Substituting $\omega\rho_w(\uu_w - \uu_i) = -\kappa_\omega\nabla\omega$ into \eqref{eq:wmomentum} and solving for $\uu_i$ yields
\begin{equation*}
  \uu_i = \uu + \rho^{-1} \kappa_\omega \nabla\omega
\end{equation*}
as in \eqref{eq:vhtclosure:uice}.
In \eqref{eq:vhtstrong:energy}, the mass flux is converted to energy flux $-L\kappa_\omega\nabla\omega$ using the latent heat $L$.

The present formulation is similar to \citet{aschwanden2011enthalpy} and approaches their model in the limit $\omega\to 0$, but is more conservative.
The formulation here uses a 3D momentum balance and conserves mass, momentum, and energy, regardless of large amplitude moisture content and presence of numerical diffusion.
We have intentionally neglected to write constitutive relations of the form $e(T,\omega,p)$ because such relationships are superfluous for the purpose of solving the equations.
They can be obtained by inverting the constitutive relations presented here and, depending on the experimental setup, may be useful for model validation.

For the purpose of determining cell Peclet numbers (a diagnostic tool and possible input to numerical stabilization methods), it is useful to write the thermal and moisture flux in terms of total energy gradient using
\begin{align*}
  \frac{\partial T}{\partial E}      & = \frac{\partial T}{\partial e}\frac{\partial e}{\partial E} \approx \frac{1}{c_i} \frac{1}{\rho} \quad\text{Cold ice} \\
  \frac{\partial \omega}{\partial E} & = \frac{\partial \omega}{\partial e}\frac{\partial e}{\partial E} \approx \frac{1}{L} \frac{1}{\rho} \quad\text{Temperate ice}
\end{align*}
so that the energy flux can be written as $-K_T \nabla E - K_\omega \nabla E$ with
\begin{align*}
  K_T      & \approx \frac{k_T}{\rho c_i} = \SI{1.15e-6}{\metre\squared\per\second} \\
  K_\omega & \approx \frac{L \kappa_\omega}{\rho L} = \SI{1.15e-7}{\metre\squared\per\second} .
\end{align*}
Note that we have neglected kinetic energy and the dependence of density on $E$ in this approximation.
For an element with streamline length $h = \SI{1}{\kilo\metre}$ and velocity $v = \SI{1}{\kilo\metre\per\year}$, the cell Peclet number is $\Peclet_h = hv / 2K = \num{1.4e4}$ for cold ice and ten times larger for temperate ice.
Numerical methods for such systems require upwinding to prevent non-physical oscillations.
Godunov's Theorem~\citep[1954, see \eg][]{leveque2002finite} states that non-oscillatory linear methods for hyperbolic equations are at most first-order accurate.
Higher order accuracy requires a nonlinear method, even if the equation being solved is linear.
Robust methods for such systems are based on finite volume methods of total variation diminishing (TVD) and total variation bounded (TVB) type, as well as discontinuous Galerkin methods with limiters when necessary~\citep{leveque2002finite,harten1983high,boris1973flux,zalesak1979fully,harten1987uniformly,liu1994weighted,jiang1996efficient,shu2003high,hesthaven2008nodal}.
In comparison, continuous finite element methods are much less robust, and most efforts to stabilize finite element methods for transport-dominated processes have used linear stabilization~\citep{brooks1982sup,hughes1989new,hughes1998variational,matthies2008stabilization}.
Attempts to use nonlinear stabilization with continuous finite element methods have not been very successful.
The best methods, according to the recent comparisons~\citep{john2007spurious,john2008spurious,john2008femtimecdr}, involve extreme restrictions on element types and suffer from difficulty in converging the nonlinear systems~\citep{mizukami1985petrov} or an algebraic construction~\citep{kuzmin2004high} that is difficult to apply with mesh anisotropy and material nonlinearity.
For simplicity, we adopt the streamline upwind Petrov-Galerkin method of \citep{brooks1982sup}, but a discontinuous Galerkin method would be a better choice for discretization of the energy equation on unstructured grids.

An additional detail appears in the finite element discretization of \eqref{eq:vhtstrong}.
The symmetric gradient of ice velocity $D\uu_i$ is needed to define the stress, but only the gradient of momentum
\begin{equation}\label{eq:vhtgradu}
  \begin{split}
    \nabla(\rho\uu) &= \uu \otimes \nabla\rho + \rho \nabla\uu \\
    &= \uu \otimes \nabla\rho + \rho \nabla (\uu_i - \rho^{-1} \kappa_\omega\nabla \omega)
  \end{split}
\end{equation}
is avialable.
Density is not an explict variable in this formulation and the definition of density involves the gradient of the explicit variables, therefore solving for $\nabla\uu_i$ in \eqref{eq:vhtgradu} produces a second order term.
This term can be evaluated using the second derivative of the basis functions or using a local projection, but neither of these methods are conveniently available in {\Dohp} at present, therefore we define $\nabla\uu_i$ with the approximation
\begin{equation*}
  \nabla(\rho\uu) \approx \uu \otimes \nabla\rho + \rho \nabla \uu_i
\end{equation*}
of \eqref{eq:vhtgradu}.

Although this simulation is run using realistic geometry on a section of the Jakobshavn Isbr{\ae} channel, not all boundary conditions necessary for a realistic simulation have been implemented in {\Dohp}.
More sophisticated lateral boundary conditions, energy conditions, and free-surface evolution are not considered in the present model.
See \citet{aschwanden2011enthalpy} for further discussion of boundary conditions for energy transport.
The output of this simulation is \emph{not} intended as a predictive model, instead it is a demonstration of the capability of the methods to handle systems that were previously not possible or too computationally expensive due to the need for short time steps.
