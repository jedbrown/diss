We consider the problem of computing a steady-state energy (in the form of temperature, moisture, and kinetic energy) field coupled to a Stokes problem with rheology depending on strain rate, pressure, temperature, and melt fraction.
This problem is most important as part of inversion for a temperature field that is compatible with the observed geometry, flow field, borehole temperature measurements, and any other field observations.
Time-stepping a transient model to steady state is extremely expensive, therefore it will never be feasible as part of the functional to be minimized by an optimization algorithm.

\subsection{Problem description}\label{ssec:vhtproblem}
Given a Lipschitz domain $\Omega$ with surface $\Gamma_s$ and time interval $(0,\tau)$, the strong form is to find total momentum, pressure, and total energy density $(\rho\uu,p,E) \in W^{1,\pfrak} \times L^2 \times H^1$ such that
\begin{subequations}\label{eq:vhtstrong}
  \begin{align}
    (\rho\uu)_t + \div (\rho\uu\otimes\uu - \eta D\uu_i + p\bm 1) - \rho \bm g &= 0 \label{eq:vhtstrong:momentum} \\
    \rho_t + \div \rho\uu &= 0 \label{eq:vhtstrong:mass} \\
    E_t + \div \left((E+p)\uu - k_T\nabla T - L (1-\omega)\frac{\rho_i}{\rho}\kappa_\omega\nabla\omega \right) - \eta D\uu_i\tcolon D\uu_i - \rho\uu\cdot\bm g &= 0 \label{eq:vhtstrong:energy}
  \end{align}
\end{subequations}
on $\Omega\otimes (0,\tau)$, with Dirichlet flow boundary conditions except at the free surface, and all Dirichlet boundary conditions for energy $E$.
These equations represent conservation of momentum, mass, and energy respectively.
Note that $\rho_t$ appears in \eqref{eq:vhtstrong:mass}, but $\rho$ is not an explicit variable and only depends on pressure when the melt fraction is positive, therefore this system is still differential algebraic.
The energy equation consists of a transport term, thermal diffusion, moisture diffusion, and heat production due to strain heating.
We switch immediately to the steady-state form of \eqref{eq:vhtstrong} in which $(\rho\uu)_t$, $\rho_t$, and $E_t$ are all zero.
Constitutive relations are needed for total density $\rho$ (\si{\kilogram\per\metre}), ice velocity $\uu_i$ (\si{\metre\per\second}), temperature $T$ (\si{\kelvin}), volumetric moisture fraction (porosity) $\omega$ (nondimensional), and viscosity $\eta$ ($\si{\pascal\second} = \si{\kilogram\per\metre\per\second}$).
In general, each constitutive relation is a function of all field variables.
The thermal conductivity $k_T$ (\si{\joule\per\metre\per\kelvin\per\second}) and hydraulic diffusivity $\kappa_\omega$ (\si{\kilogram\metre\per\second}) are taken to be constant because experimental data are sparse, but this assumption is in no way critical.

The constitutive relations for temperature and moisture fraction are usually defined piecewise.
It is preferable for the convergence of Newton methods~\citep[\cf][]{gropp2000globalized} to have a discretization with $C^1$ continuity and it is simpler for manufactured solutions if the constitutive relation has a global (\ie not piecewise) definition in terms of analytic functions.
To achieve this, we decompose the function $\mathcal{S}(x) = x$ into two globally smooth parts
\begin{align*}
  %\mathcal S_\delta^-(x) &= \frac{x}{2} \left(1 - \erf \frac{x}{\sqrt 2 a} \right)  -\frac{a}{\sqrt{2\pi}} \exp{\frac{-x^{2}}{2 a^{2}}} \\
  %\mathcal S_\delta^+(x) &= \frac{x}{2} \left(1 + \erf \frac{x}{\sqrt 2 a} \right) + \frac{a}{\sqrt{2\pi}} \exp{\frac{-x^{2}}{2 a^{2}}} \\
  \mathcal S_\delta^-(x) &= \frac{x}{2} - \frac{x}{2} \erf \frac{x}{\sqrt 2 \delta} - \frac{\delta}{\sqrt{2\pi}} \exp{\frac{-x^{2}}{2 \delta^{2}}} \\
  \mathcal S_\delta^+(x) &= \frac{x}{2} + \frac{x}{2} \erf \frac{x}{\sqrt 2 \delta} + \frac{\delta}{\sqrt{2\pi}} \exp{\frac{-x^{2}}{2 \delta^{2}}}
\end{align*}
which satisfy $\mathcal S_\delta^- (x) < 0$, $\mathcal S_\delta^+(x) > 0$ and $\mathcal S_\delta^-(x) + S_\delta^+(x) = x$.
These functions arise from integrating the error function with standard deviation $\delta$.
Taking $\delta\to 0$ recovers the piecewise linear decomposition $\mathcal S_0^-(x) = \min(x,0)$, $\mathcal S_0^+(x) = \max(x,0)$.
In applications where a single global function is not important, the decomposition $\mathcal S^\pm$ could be defined using a spline which would reduce the high computational cost of evaluating $\erf$.
This decomposition will be used to separate internal energy into thermal and melt contributions.

For convenience in defining constitutive relations, we introduce specific internal energy $e$ (\si{\joule\per\kilogram}) which is related to total and kinetic energy through
\begin{equation*}
  E = \rho e + \half (1-\omega) \rho_i \abs{\uu_i}^2 + \half \omega \rho_w \abs{\uu_w}^2
\end{equation*}
which we approximate as
\begin{equation}\label{eq:intenergy}
  E = \rho e +  \frac{1}{2\rho} \abs{\rho\uu}^2 .
\end{equation}
This approximation may be violated, for example, at moderate porosity when the velocity of the melt fraction is very high compared to the bulk velocity, such as in an actively draining moulin.
In such circumstances, it is likely unavoidable to add additional variables for water momentum and energy.
Consistent with exact incompressibility, the internal energy is independent of pressure even though observable quantities like temperature and moisture fraction are dependent on pressure.
Removing this assumption would produce acoustic waves and a conservative formulation would require that density be an explicit degree of freedom.
When combined with the closure for $\rho(p,e)$, shown below, \eqref{eq:intenergy} requires solving an implicit equation involving the decomposition $\mathcal S^\pm$.
This implicit equation can always be reduced to one dimension and can be solved explicitly for some definitions of $\mathcal S^\pm$.
For some purposes, it is acceptable to simply take $\rho \approx \rho_i$.
%, therefore we use the density of pure ice $\rho \approx \rho_i$ and the ice velocity $\uu_i = \uu$ (valid for $\omega \ll 1$) in this equaiton \emph{only}.
The convective contributions to momentum balance $\rho\uu\otimes\uu$ and kinetic energy $\frac{1}{2\rho}\abs{\rho\uu}^2$ have vanishing influence in glaciology, but are easy to accommodate so we keep them for completeness.
An alternative would be to discretize using $\rho$ instead of $p$ as the independent variable, but near incompressibility and the variation due to moisture content causes the resulting system to be extremely ill-conditioned.

The closures for \eqref{eq:vhtstrong} are
\begin{subequations}\label{eq:vhtclosure}
  \begin{align}
    \rho(p,e)        & = \big(1-\omega(p,e) \big) \rho_i + \omega(p,e) \rho_w                            \\
    \uu_i(\uu,p,e)   & = \uu + \rho(p,e)^{-1} \kappa_\omega \nabla\omega(p,e) \label{eq:vhtclosure:uice} \\
    T(p,e)           & = T_0 + \frac{e_m(p) + \mathcal S_\delta^-\big(e - e_m(p)\big)}{c_i} \\
    \omega(p,e)      & = \frac{\rho_i \mathcal S_\delta^+\big(e - e_m(p)\big)}{\rho_w L - (\rho_w-\rho_i)\mathcal S_\delta^+\big(e - e_m(p)\big)} \\
    \eta(\gamma,p,e) & = B(p,e)\left(\epsilon^2 + \frac{\gamma}{\gamma_0} \right)^{\frac{\mathfrak{p}-2}{2}}
  \end{align}
\end{subequations}
with second invariant $\gamma = \half D\uu\tcolon D\uu$ and the additional constitutive relations
\begin{align*}
  T_m(p)   & = T_3 - \beta_{CC} p          \\
  e_m(p)   & = c_i \big(T_m(p) - T_0 \big) \\
  T^*(p,e) & = T(p,e) - T_m(p) + T_3       \\
  B(p,E)   & = B_0 \exp \left( \frac{Q - pV}{\nfrak R T^*(p,e)} - \frac{Q}{\nfrak R T_0} \right) \big(1 + B_\omega \omega(p,E) \big)^{-1/\nfrak}
\end{align*}
with physical constants given in \tabref{tab:vhtconst}.
Since dimensional units are used here unlike in earlier sections, the strain rate regularization $\epsilon$ is now a fraction of $\gamma_0$ which is the second invariant of a reference strain rate.
The Arrhenius relation is normalized to zero pressure and a reference temperature $T_0$.
This formulation is not conventional in glaciology, but permits more intuitive understanding of parameters because they are no longer sensitive to the power law exponent $\pfrak$ and large exponential terms.

\begin{table}
  \centering
  \begin{tabular}{clll}
    \toprule
    Symbol          & Value                                         & Description                                                       \\
    \midrule
    $c_i$           & \SI{2009}{\joule\per\kilogram\per\kelvin}     & Specific heat capacity of ice                                     \\
    %$c_w$          & \SI{4170}{\joule\per\kilogram\per\kelvin}     & Specific heat capacity of water                                   \\
    $k_T$           & \SI{2.1}{\watt\per\metre\per\kelvin}          & Thermal conductivity of ice                                       \\
    $\rho_i$        & \SI{910}{\kilogram\per\metre\cubed}           & Density of ice                                                    \\
    $\rho_w$        & \SI{1000}{\kilogram\per\metre\cubed}          & Density of liquid water                                           \\
    $L$             & \SI{3.34e5}{\joule\per\kilogram}              & Latent heat of fusion                                             \\
    $g$             & \SI{9.81}{\metre\per\second\squared}          & Gravitational acceleration                                        \\
    $\kappa_\omega$ & \SI{1.045e-4}{\kilogram\per\metre\per\second} & Hydraulic diffusivity of ice                                      \\
    %$K_w$          & \SI{1.045e-4}{\kilogram\per\metre\per\second} & Temperature ice diffusivity
    $\beta_{CC}$    & \SI{7.9e-8}{\kelvin\per\pascal}               & Clausius-Capeyron gradient                                        \\
    $T_3$           & \SI{273.15}{\kelvin}                          & Triple point of water                                             \\
    \midrule
    $\gamma_0$      & $\half (\SI{1e-10}{\per\second})^2$           & Second invariant of reference strain rate                         \\
    $T_0$           & \SI{260}{\kelvin}                             & Reference temperature                          \\
    $B_0$           & \SI{8.56e14}{\pascal\second}                  & Viscosity at reference strain rate and temperature                \\
    $Q$             & \SI{6.0e4}{\joule\per\mole}                   & Activation energy for creep                                       \\
    $V$             & \SI{-13.e-6}{\metre\cubed\per\mole}           & Activation volume for creep                                       \\
    $R$             & \SI{8.31441}{\joule\per\mole\per\kelvin}      & Ideal gas constant                                                \\
    $B_\omega$      & 181.25                                        & Influence of water content on viscosity \citep{greve2009dynamics} \\
    \bottomrule
  \end{tabular}
  \caption{Physical constants used for the viscous heat transport problem.
    The same constants are used in \citet{aschwanden2011enthalpy}.}\label{tab:vhtconst}
\end{table}

The definitions of the moisture flux in \eqref{eq:vhtstrong:energy} and ice velocity $\uu_i$ require further explanation.
The total momentum can be defined in terms of constituent momenta as
\begin{equation}\label{eq:wmomentum}
  \begin{split}
    \rho\uu & = (1-\omega) \rho_i \uu_i + \omega \rho_w \uu_w                             \\
            & = (1-\omega) \rho_i \uu_i + \omega \rho_w u_i + \omega\rho_w(\uu_w - \uu_i) \\
            & = \rho \uu_i + \omega\rho_w(\uu_w - \uu_i) .
  \end{split}
\end{equation}
The second term is the momentum of the moisture content in the reference frame of the ice.
The mass flux of the moisture is $-\kappa_\omega\nabla\omega$ which is also the momentum density.
That is, the integral of $-\kappa_\omega\nabla\omega$ over a surface element is the mass flux (\si{\kilogram\per\second}) through that surface, while the integral over a volume element is the momentum (\si{\kilogram\metre\per\second}) of that volume.
Substituting $\omega\rho_w(\uu_w - \uu_i) = -\kappa_\omega\nabla\omega$ into \eqref{eq:wmomentum} and solving for $\uu_i$ yields
\begin{equation*}
  \uu_i = \uu + \rho^{-1} \kappa_\omega \nabla\omega
\end{equation*}
as in \eqref{eq:vhtclosure:uice}.
In \eqref{eq:vhtstrong:energy}, we need the energy flux in the reference frame of the total velocity.
Starting from the mass flux in the reference frame of total velocity,
\begin{equation}\label{eq:vhtmomflux}
  \begin{split}
    \omega\rho_w(\uu_w - \uu) & = \omega \rho_w (\uu_w - \uu_i) + \omega\rho_w(\uu_i - \uu)                             \\
                              & = -\kappa_\omega\nabla\omega + \frac{\omega\rho_\omega}{\rho}\kappa_\omega\nabla \omega \\
                              & = - \left( 1 - \frac{\omega\rho_\omega}{\rho} \right) \kappa_\omega \nabla \omega \\
                              & = - (1-\omega)\frac{\rho_i}{\rho}\kappa_\omega\nabla\omega
  \end{split}
\end{equation}
where the moisture flux and \eqref{eq:vhtclosure:uice} was used on the second line.
The mass flux in \eqref{eq:vhtmomflux} is converted to energy flux by multiplying by the latent heat $L$ to produce the moisture flux appearing in \eqref{eq:vhtstrong:energy}.
The use of constant hydraulic conductivity $\kappa_\omega$ is a poor approximation for large amplitude $\omega$ since intraglacial conduits form for higher melt fractions, thus conductivity becomes nearly infinite.
For such cases, it would likely be better to use a Darcy-type constitutive relation accommodating the gravitational contribution and with conductivity dependent on moisture, perhaps of the form $\kappa(\omega) = \kappa_0 \exp \frac{\omega}{\omega_0}$ where $\kappa_0$ is the conductivity for vanishing moisture fraction and $\omega_0$ is a characteristic melt fraction on the order of \SI{1}{\percent}.

The volumetric flux $p\bm u$ appearing in \eqref{eq:vhtstrong:energy} could have been written as part of a single heating term.
For single-phase compressible flows, the heat production can be written $(\eta D\uu - p\bm 1)\tcolon \grad \uu$ from classical definitions of work~\citep[\eg][]{hutter2004continuum}, plus a kinetic energy contribution $-\uu\cdot\nabla p$.
These can be manipulated as
\begin{equation}\label{eq:heatdiss}
  \begin{split}
    (\eta D\uu - p\bm 1)\tcolon \grad \uu - \uu\cdot\grad p &= \eta D\uu\tcolon \grad\uu - p\div\uu - \uu\cdot\grad p \\
    &= \eta D\uu\tcolon D\uu - \div (p\uu)
  \end{split}
\end{equation}
which is the form appearing in \eqref{eq:vhtstrong:energy}.
Although both expressions are equivalent for the continuous equations, the discrete equations are different because numerical solutions only satisfy partial differential equations weakly.
Only the latter formulation is conservative.
This formulation is also more amenable to discretization and is mandatory for stability when density is dependent on pressure (as it is here when melt content is available) and diffusion is poorly resolved on the mesh.
The first term in \eqref{eq:heatdiss} is non-reversible dissipation while the second can be recovered through volume change and thus does not affect the global entropy.
The ice velocity $\uu_i$ is used in \eqref{eq:vhtstrong:energy} because the melt fraction is assumed to be able to move through the ice matrix without viscous dissipation.
Perhaps this is a reasonable assumption because the viscosity of ice is \num{1e15} larger than water, but for relatively stagnant ice with high moisture velocity, it is likely to be significant and the equations can be augmented with an additional diffusive term.

The present formulation is similar to \citet{aschwanden2011enthalpy} and approaches their model in the limit $\omega\to 0$, but is more conservative.
The formulation here uses a 3D momentum balance and conserves mass, momentum, and energy, regardless of large amplitude moisture content and presence of numerical diffusion.
We have intentionally neglected to write constitutive relations of the form $e(T,\omega,p)$ because such relationships are superfluous for the purpose of solving the equations.
They can be obtained by inverting the constitutive relations presented here and, depending on the experimental setup, may be useful for model validation.

For the purpose of determining cell Peclet numbers (a diagnostic tool and possible input to numerical stabilization methods), it is useful to write the thermal and moisture fluxes using
\begin{align*}
  \frac{\partial T}{\partial E}      & = \frac{\partial T}{\partial e}\frac{\partial e}{\partial E} \approx \frac{1}{c_i} \frac{1}{\rho} \quad\text{Cold ice} \\
  \frac{\partial \omega}{\partial E} & = \frac{\partial \omega}{\partial e}\frac{\partial e}{\partial E} \approx \frac{1}{L} \frac{1}{\rho} \quad\text{Temperate ice}
\end{align*}
so that the diffusive energy flux driven by energy gradient can be written as $-K_T \nabla E - K_\omega \nabla E$ with
\begin{align*}
  K_T      & \approx \frac{k_T}{\rho c_i} = \SI{1.15e-6}{\metre\squared\per\second} \\
  K_\omega & \approx \frac{L \kappa_\omega}{\rho L} = \SI{1.15e-7}{\metre\squared\per\second} .
\end{align*}
Note that we have neglected kinetic energy and density dependence in this approximation.
For an element with streamline length $h = \SI{1}{\kilo\metre}$ and velocity $v = \SI{1}{\kilo\metre\per\year}$, the cell Peclet number is $\Peclet_h = hv / K = \num{2.8e4}$ for cold ice and ten times larger for temperate ice.
Numerical methods for such systems require upwinding to prevent non-physical oscillations.
Godunov's Theorem~\citep[1954, see \eg][]{leveque2002finite} states that non-oscillatory linear methods for hyperbolic equations are at most first-order accurate.
Higher order accuracy requires a nonlinear method, even if the equation being solved is linear.
Robust methods for such systems are based on finite volume methods of total variation diminishing (TVD) and total variation bounded (TVB) type, as well as discontinuous Galerkin methods with limiters when necessary~\citep{leveque2002finite,harten1983high,boris1973flux,zalesak1979fully,harten1987uniformly,liu1994weighted,jiang1996efficient,shu2003high,hesthaven2008nodal}.
In comparison, continuous finite element methods are much less robust, and most efforts to stabilize finite element methods for transport-dominated processes have used linear stabilization~\citep{brooks1982sup,hughes1989new,hughes1998variational,matthies2008stabilization}.
Attempts to use nonlinear stabilization with continuous finite element methods have not been very successful.
The best methods, according to the recent comparisons~\citep{john2007spurious,john2008spurious,john2008femtimecdr}, involve extreme restrictions on element types and suffer from difficulty in converging the nonlinear systems~\citep{mizukami1985petrov} or an algebraic construction~\citep{kuzmin2004high} that is difficult to apply with mesh anisotropy and material nonlinearity.
For simplicity, we adopt the streamline upwind Petrov-Galerkin method of \citep{brooks1982sup}, but a discontinuous Galerkin method would be a better choice for discretization of the energy equation on unstructured grids.

An additional detail appears in the finite element discretization of \eqref{eq:vhtstrong}.
The symmetric gradient of ice velocity $D\uu_i$ is needed to define the stress, but only the gradient of momentum
\begin{equation}\label{eq:vhtgradu}
  \begin{split}
    \nabla(\rho\uu) &= \uu \otimes \nabla\rho + \rho \nabla\uu \\
    &= \uu \otimes \nabla\rho + \rho \nabla (\uu_i - \rho^{-1} \kappa_\omega\nabla \omega)
  \end{split}
\end{equation}
is avialable.
Density is not an explict variable in this formulation and the definition of density involves the gradient of the explicit variables, therefore solving for $\nabla\uu_i$ in \eqref{eq:vhtgradu} produces a second order term.
This term can be evaluated using the second derivative of the basis functions or using a local projection, but neither of these methods are conveniently available in {\Dohp} at present, therefore we approximate $\nabla\uu_i$ using
\begin{equation*}
  \nabla(\rho\uu) \approx \uu \otimes \nabla\rho + \rho \nabla \uu_i
\end{equation*}
of \eqref{eq:vhtgradu}.

Although this simulation is run using realistic geometry on a section of the Jakobshavn Isbr{\ae} channel, not all boundary conditions necessary for a realistic simulation have been implemented in {\Dohp}.
More sophisticated lateral boundary conditions, energy conditions, and free-surface evolution are not considered in the present model.
See \citet{aschwanden2011enthalpy} for further discussion of boundary conditions for energy transport.
The output of this simulation is \emph{not} intended as a predictive model, instead it is a demonstration of the capability of the methods to handle systems that were previously not possible or too computationally expensive due to the need for short time steps.

\subsection{Verification}\label{ssec:vhtverif}
To verify the correctness of the implementation, we consider a manufactured solution with rich structure and choose a parameter range to activate all the terms in \eqref{eq:vhtstrong} and constitutive relations.
Instead of the physical parameters in \tabref{tab:vhtconst}, we take all parameters to be of order one, with solid and melt densities of 1 and 2 respectively.
For this problem, the melt fraction rises as high as \SI{28}{\percent} and temperature ranges \SI{11}{\percent} of its absolute value.
Since the activation volume $V$ and Clausius-Capeyron gradient $\beta_{CC}$ are relatively large, the pressure plays a significant, and occasionally dominant role in defining the temperature and the material rheology.
The large moisture content and large density contrast also increase the strength of the nonlinearity.

Due to the non-uniform flow and various constitutive nonlinearities, nondimensional numbers are spatially variable.
The Reynolds number ranges up to about \num{2.4}, the Peclet number ranges up to \num{5.3}, and the Prandtl number ranges from \num{0.6} to 1.
A computed solution, accurate to \SI{0.5}{\percent} is shown in \figref{fig:vhtexact}.
As usual, this manufactured solution is not physically realizable, but it excercises all the terms.

\begin{figure}
  \centering\includegraphics[width=\textwidth]{visit0020}
  \caption{Manufactured solution as computed using a $\Qk 3 - \Qk 2 - \Qk 3$ finite element discretization on a $12\times 12\times 12$ mesh.
    This solution is accurate to 4 digits for momentum and energy, with two digits of accuracy for pressure.
    The gradients are accurate to 3 digits for momentum and energy, with \SI{3}{\percent} error for pressure gradient.}\label{fig:vhtexact}
\end{figure}

We consider norms for a $\Qk 3 - \Qk 2 - \Qk 3$ approximation under $h$-refinement.
Due to the direct appearance of pressure in the equations, we cannot expect to realize fourth order convergence, at least not for the energy equation.
Figure~\ref{fig:vhtrefine} shows the observed convergence behavior in which energy clearly converges with only third order accuracy.
I do not have an explanation for why the convergence eventually stagnates, starting with the energy equation.
It could be an artifact of the manufactured solution process (perhaps rectifiable using more accurate quadrature for the forcing term), other quadrature errors due to nonlinearity, or stability of the continuum equations or discretization.

\begin{figure}
  \centering\includegraphics{vhtdisc}
  \caption{Convergence rates for $\Qk 3 - \Qk 2 - \Qk 3$ under $h$-refinement.}\label{fig:vhtrefine}
  % ./vhtconvergence.py --plot -o vhtdisc.pdf
\end{figure}

The nonlinear solver converges quadratically as seen in \tabref{tab:vhtsnes}.
\begin{table}
  \centering
  \begin{tabular}{lllll}
    \toprule
    Iteration & Mass         & Momentum     & Energy       & Total        \\
    \midrule
    0         & 2.142762e-01 & 1.431024e+01 & 2.742861e+01 & 3.093796e+01 \\
    1         & 1.086178e-05 & 8.386431e+00 & 8.412471e+00 & 1.187863e+01 \\
    2         & 2.928430e-06 & 4.103421e+00 & 3.579857e+00 & 5.445497e+00 \\
    3         & 1.744093e-06 & 3.059956e+00 & 6.853340e-01 & 3.135764e+00 \\
    4         & 8.688964e-07 & 1.891518e+00 & 2.980380e-01 & 1.914854e+00 \\
    5         & 4.952597e-07 & 6.852763e-01 & 7.214378e-02 & 6.890634e-01 \\
    6         & 1.430063e-07 & 4.827890e-02 & 6.381075e-03 & 4.869877e-02 \\
    7         & 1.057706e-08 & 3.257086e-05 & 3.007905e-05 & 4.433521e-05 \\
    8         & 1.759267e-11 & 4.249319e-10 & 1.387815e-10 & 4.473666e-10 \\
    \bottomrule
  \end{tabular}
  \caption{Nonlinear convergence rates.}\label{tab:vhtsnes}
\end{table}

Numerical experiments with nonlinear solver convergence and pseudo-transient continuation~\citep{coffey2003ptc,kelley1998cap} indicates that this system does not always have steady-state solutions, and when steady-state solutions exist, they may be non-unique.
It would be interesting to explore these uniqueness properties using bifurcation techniques such as those in \citet{allgower2003inc}.
However, these phenomena have not been observed in parameter ranges that are realistic for ice flow, so we do not explore them further.
Note that such nonlinear effects do occur in glaciology, but generally involve sliding and/or geometry change.
With thermally-induced density variation, gravity, and a boundary heat source, there are not steady-states above a critical Rayleigh number.
This is seen in mantle convection and other fields, for which \eqref{eq:vhtstrong} is also valid, therefore it is no surprise that steady-states are not present.

For Dirichlet problems, pressure is only determined up to a constant.
As far as the solver is concerned, this is easy to handle by applying the Krylov iteration with null space removed and using a preconditioner that is tolerant of the one dimensional null space.
Most preconditioners other than direct solvers are suitable, and the methods of \secref{sec:multiphysics:fieldsplit} perform fine as long as inner solvers (if used) are also informed of the null space.
However, unlike for linear and power-law Stokes problems, the absolute value of pressure affects the other variables.
Thus there is a one-parameter family of solutions in which all variables change rather than only the pressure.
Additionally, it is possible to compute a negative pressure during the nonlinear solve.
This is most common at higher viscosity because the forcing terms in the manufactured solutions become huge, thus producing extreme values of pressure.
Negative pressures produce feedback through the activation volume $V$ to generate exponentially large viscosity.
