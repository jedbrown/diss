Jakobshavn Isbr{\ae} is representative of many similar outlet glaciers which control the response of the Greenland and Antarctic ice sheets to changing climate.
Understanding the effect of changing boundary conditions, especially ocean conditions near the grounding line and subglacial processes influenced by surface melt, is critical for predictive modeling of ice sheets.


\section{Meshing}
For the surface, we use the digital elevation model compiled by \citet{motyka2010volume} using aerial photography taken 24 July 1985 and ground control points resurveyed during our field campaigns~\todo{cite?}.
The bedrock location was provided by CReSIS~\todo{cite?}.
A smooth geometric model based on an unstructured triangular mesh was constructed from these rasters using MOAB~\citep{moab} and other tools being developed as part of the MeshKit project~\citep{meshkit}.
This unstructured triangular mesh was decimated to reduce size while preserving features.
The boundary of the region of interest was described with a polygon in the map plane.
The bed, surface, and lateral cut defines three geometric surfaces and a volume.
Mesh generation involves two phases, unstructured quadrilateral meshing in the horizontal followed by graded sweeping in the vertical.
Bed slope was used as a refinement indicator for the horizontal, the quadrilateral surface mesh was generated using the CUBIT Adaptive Meshing Algorithm Library~\citep{blacker1994cmg}.
This mesh was swept to create a hexahedral mesh, but the quadrilateral elements on the surfaces were still stored, along with association to the geometric model.
In the future, this geometric model will be used to improve geometric fidelity and define high-order slip conditions that allow the mesh to move along the geometry.
Additionally, we intend to further improve mesh quality, especially in the vicinity of steep bed and surface slopes, by solving elliptic smoothing equations~\citep{liseikin2009grid} (e.g. Laplace-Beltrami using the methods of \citet{hansen2007unstructured,berndt2008efficient}, though we are investigating a modification of the recently proposed mesh motion formulations based on Monge-Kantorovich optimization~\citet{delzanno2008optimal,chacon2011robust} that would be favorable for evolution problems).

The mesh was read by the analysis code and high-order function spaces for velocity and pressure were defined on it according to runtime parameters.
Boundary conditions were defined using the three tagged boundary sets, with a free surface, no-slip at the bed, and lateral velocities given by the shallow ice approximation.
Due to surface and bed roughness, these lateral boundary conditions are noisier than desired, but the noise can be seen to dissipate in one to two ice thicknesses and thus has little influence on the flow in the vicinity of the ice stream.
% as
% \begin{equation}\label{eq:surfacevel}
%   u(x,y,z) = u_{\text{surf}}(x,y) \left( \frac{h-z}{h-b} \right)^{1/(\mathfrak n+1)}
% \end{equation}
% where $\nfrak = 3$ is the Glen exponent, which is consistent with the velocity profile expected from the shallow ice approximation.
% The surface velocity map $u_{\text{surf}}(x,y)$ was obtained from Ian Joughin~\todo{cite}.

\todo{Explain problem and background}

\begin{figure}
  \centering\includegraphics[width=\textwidth]{jako-clip0-q3}
  \caption{Velocity color plot of part of Jakobshavn Isbr{\ae} using $Q_3$ velocity elements and a graded mesh. This is only a placeholder because of incorrect boundary conditions (need to sample from Ian Joughin's files which are in a non-standard format and projection so GDAL does not read them.)}\label{fig:jako-clip}
\end{figure}

% Boundary sets:
% 1: Bottom
% 2: Surface
% 3: Cuts at horizontal margins
% src/fs/tests/stokes -dmesh_in ~/dl/he6.h5m -snes_max_it 1 -snes_monitor_short -ksp_type preonly -pc_type fieldsplit -pc_fieldsplit_type schur -pc_fieldsplit_real_diagonal -fieldsplit_p_ksp_monitor_short -fieldsplit_p_ksp_type cg -fieldsplit_u_ksp_converged_reason -stokes_Ap_mat_type sbaij -const_bdeg 3 -pressure_codim 1 -gravity -1. -dirichlet 1,3 -viewdhm -dmesh_intermediate_adjacencies -stokes_case jako -jako_surface_velocity foo

% Joughin's velocity data
% Lat: 70
% Lon: -45
% Pixel size: 100m by 100m
% False Easting: -217.75e3
% False Northing: -2302.00e3
% No Data: -2.0e9
% Grid is EPSG:3413 (I think), could be EPSG:3411
%
% Grid the bed elevations, see
%
% gdal_grid -a_srs utm22n.wkt -a 'nearest:radius1=0.0:radius2=0.0:angle=0.0:nodata=-9999' -outsize 400 400 -of GTiff -ot Float64 -txe 545000 602000 -tye 7700000 7657000 -l jak_bed_elevation_data_utm jak_bed_elevation_data_utm.vrt jak_bed_elevation_data_utm_400_nearest.tiff
%
%  gdal_grid -a 'invdist:power=2.0:smoothing=1.0:nodata=-9999' -outsize 400 400 -of GTiff -ot Float64 -txe 545000 602000 -tye 7700000 7657000 -l jak_bed_elevation_data_utm jak_bed_elevation_data_utm.vrt jak_bed_elevation_data_utm_400.tiff
%
% OGRErr OGRSpatialReference::SetStereographic(double dfCenterLat,double dfCenterLong,double dfScale,double dfFalseEasting,double dfFalseNorthing)

% http://www.gdal.org/gdal_grid.html#gdal_grid_csv
% 
