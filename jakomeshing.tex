For the surface, we use the digital elevation model compiled by \citet{motyka2010volume} using aerial photography taken 24 July 1985 and ground control points resurveyed during our field campaigns in summer 2007 and 2008.
The bedrock location was provided by CReSIS~\cite{plummer2011highres}.
A smooth geometric model based on an unstructured triangular mesh was constructed from these rasters using MOAB~\citep{moab} and other tools being developed as part of the MeshKit project~\citep{meshkit}.
This unstructured triangular mesh was decimated to reduce size while preserving features.
The boundary of the region of interest was described with a polygon in the map plane.
Figure~\ref{fig:jakoregion} shows the decimated and meshed regions.

\begin{figure}
  \centering\includegraphics[width=\textwidth]{jakoregion}
  \caption{Regions of interest for Jakobshavn Isbr{\ae}.
    The outer black line marks the area on which bed and surface measurements were accurate enough to perform decimation.
    The inner (thick) black line marks the meshed region.}\label{fig:jakoregion}
\end{figure}

The bed, surface, and lateral cut defines three geometric surfaces and a volume which is shown in \figref{fig:jakotransparent} resting on the decimated geometric model for the bed.
Mesh generation involves two phases, unstructured quadrilateral meshing in the horizontal followed by graded sweeping in the vertical.
Bed slope was used as a refinement indicator for the horizontal, the quadrilateral surface mesh was generated using the CUBIT Adaptive Meshing Algorithm Library~\citep{blacker1994cmg}.
This mesh was swept to create a hexahedral mesh, but the quadrilateral elements on the surfaces were still stored, along with association to the geometric model.
In the future, this geometric model will be used to improve geometric fidelity and define high-order slip conditions that allow the mesh to move along the geometry.
Additionally, we intend to further improve mesh quality, especially in the vicinity of steep bed and surface slopes, by solving elliptic smoothing equations~\citep{liseikin2009grid} (e.g. Laplace-Beltrami using the methods of \citet{chacon2006fully,hansen2007unstructured,berndt2008efficient}, though we are investigating a modification of the recently proposed mesh motion formulations based on Monge-Kantorovich optimization~\citet{delzanno2008optimal,chacon2011robust} that would be favorable for evolution problems).
See \citet{budd2009adaptivity} for a recent survey of moving mesh methods.

\begin{figure}
  \centering\includegraphics[width=\textwidth]{jakotransparent}
  \caption{The volume to be meshed resting on the geometric model for the bed.}\label{fig:jakotransparent}
\end{figure}

% Boundary sets:
% 1: Bottom
% 2: Surface
% 3: Cuts at horizontal margins
% src/fs/tests/stokes -dmesh_in ~/dl/he6.h5m -snes_max_it 1 -snes_monitor_short -ksp_type preonly -pc_type fieldsplit -pc_fieldsplit_type schur -pc_fieldsplit_real_diagonal -fieldsplit_p_ksp_monitor_short -fieldsplit_p_ksp_type cg -fieldsplit_u_ksp_converged_reason -stokes_Ap_mat_type sbaij -const_bdeg 3 -pressure_codim 1 -gravity -1. -dirichlet 1,3 -viewdhm -dmesh_intermediate_adjacencies -stokes_case jako -jako_surface_velocity foo

% Joughin's velocity data
% Lat: 70
% Lon: -45
% Pixel size: 100m by 100m
% False Easting: -217.75e3
% False Northing: -2302.00e3
% No Data: -2.0e9
% Grid is EPSG:3413 (I think), could be EPSG:3411
%
% Grid the bed elevations, see
%
% gdal_grid -a_srs utm22n.wkt -a 'nearest:radius1=0.0:radius2=0.0:angle=0.0:nodata=-9999' -outsize 400 400 -of GTiff -ot Float64 -txe 545000 602000 -tye 7700000 7657000 -l jak_bed_elevation_data_utm jak_bed_elevation_data_utm.vrt jak_bed_elevation_data_utm_400_nearest.tiff
%
%  gdal_grid -a 'invdist:power=2.0:smoothing=1.0:nodata=-9999' -outsize 400 400 -of GTiff -ot Float64 -txe 545000 602000 -tye 7700000 7657000 -l jak_bed_elevation_data_utm jak_bed_elevation_data_utm.vrt jak_bed_elevation_data_utm_400.tiff
%
% OGRErr OGRSpatialReference::SetStereographic(double dfCenterLat,double dfCenterLong,double dfScale,double dfFalseEasting,double dfFalseNorthing)

% http://www.gdal.org/gdal_grid.html#gdal_grid_csv
% 
