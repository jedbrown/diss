A large amount of scientific effort worldwide is focused on understanding and predicting the advance and consequences of climate change, due to its potentially disastrous affects on human society.
Numerical models are playing an ever-increasing role in the analysis of complex processes, but current modeling approaches are limiting the scope of problems that can be addressed.
With the ever-increasing complexity, it is difficult to verify correctness of the implementation, assess accuracy of the simulation, or distinguish between numerical and modeling errors.
The established strategies for model coupling, while generally thought to be necessary in order to manage complexity, cause significant stability and accuracy problems.
Efficiency of nonlinear solvers represent a further obstacle to high resolution and advanced analysis techniques such as optimization, uncertainty quantification, and stability analysis.
Present implementations also tend to use low-order discretizations which poorly utilize emerging hardware, are low accuracy, and cause numerical artifacts in some cases.
We address algorithmic and software issues for tightly-coupled multiphysics, efficient nonlinear solvers, throughput on current and emerging architectures, and high-order discretizations.
Several discretization challenges for the simulation of land ice are examined, with identification of methods for their resolution.
Finally, the methods are applied to a section of Jakobshavn Isbr{\ae}'s ice stream.
